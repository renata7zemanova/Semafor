% Soubory musí být v kódování, které je nastaveno v příkazu \usepackage[...]{inputenc}

\documentclass[%        Základní nastavení
  %draft,    				  % Testovací překlad
  12pt,       				% Velikost základního písma je 12 bodů
	t,                  % obsah slajdů bude vždy začínat od shora (nebude vertikálně centrovaný)
	aspectratio=1610,   % poměr stran bude 16:10 (všechny projektory v učebnách na Technické 12 Brno),
	                    % další volby jsou 43, 149, 169, 54, 32.
	unicode,						% Záložky a informace budou v kódování unicode
]{beamer}				    	% Dokument třídy 'zpráva', vhodná pro sazbu závěrečných prací s kapitolami
%\usepackage{etex}

\usepackage[utf8]		  % Kódování zdrojových souborů je v UTF-8
	{inputenc}					% Balíček pro nastavení kódování zdrojových souborů
	
\usepackage{graphicx} % Balíček 'graphicx' pro vkládání obrázků
											% Nutné pro vložení logotypů školy a fakulty

\usepackage[          % Balíček 'acronym' pro sazby zkratek a symbolů
	nohyperlinks				% Nebudou tvořeny hypertextové odkazy do seznamu zkratek
]{acronym}						
											% Nutné pro použití prostředí 'acronym' balíčku 'thesis'

%% Balíček hyperref je volán třídou beamer automaticky, proto není třeba následujícího kódu:
%\usepackage[
%	breaklinks=true,		% Hypertextové odkazy mohou obsahovat zalomení řádku
%	hypertexnames=false % Názvy hypertextových odkazů budou tvořeny
%											% nezávisle na názvech TeXu
%]{hyperref}						% Balíček 'hyperref' pro sazbu hypertextových odkazů
%											% Nutné pro použití příkazu 'nastavenipdf' balíčku 'thesis'

\usepackage{cmap} 		% Balíček cmap zajišťuje, že PDF vytvořené `pdflatexem' je
											% plně "prohledávatelné" a "kopírovatelné"

%\usepackage{upgreek}	% Balíček pro sazbu stojatých řeckých písmem
											%% např. stojaté pí: \uppi
											%% např. stojaté mí: \upmu (použitelné třeba v mikrometrech)
											%% pozor, grafická nekompatibilita s fonty typu Computer Modern!

%\usepackage{amsmath} %balíček pro sabu náročnější matematiky

\usepackage{booktabs} % Balíček, který umožňuje v tabulce používat
                      % příkazy \toprule, \midrule, \bottomrule


%%%%%%%%%%%%%%%%%%%%%%%%%%%%%%%%%%%%%%%%%%%%%%%%%%%%%%%%%%%%%%%%%
%%%%%%      Definice informací o dokumentu             %%%%%%%%%%
%%%%%%%%%%%%%%%%%%%%%%%%%%%%%%%%%%%%%%%%%%%%%%%%%%%%%%%%%%%%%%%%%

\input{nastaveni}      % v tomto souboru doplňte údaje o sobě, o názvu práce...
                       % (tento soubor je sdílený s textem práce)

%%%%%%%%%%%%%%%%%%%%%%%%%%%%%%%%%%%%%%%%%%%%%%%%%%%%%%%%%%%%%%%%%%%%%%%%

%%%%%%%%%%%%%%%%%%%%%%%%%%%%%%%%%%%%%%%%%%%%%%%%%%%%%%%%%%%%%%%%%%%%%%%%
%%%%%%     Nastavení polí ve Vlastnostech dokumentu PDF      %%%%%%%%%%%
%%%%%%%%%%%%%%%%%%%%%%%%%%%%%%%%%%%%%%%%%%%%%%%%%%%%%%%%%%%%%%%%%%%%%%%%
%% Při vloženém balíčku 'hyperref' lze použít příkaz '\pdfsettings'
\pdfsettings
%  Nastavení polí je možné provést také ručně příkazem:
%\hypersetup{
%  pdftitle={Název studentské práce},    	% Pole 'Document Title'
%  pdfauthor={Autor studenstké práce},   	% Pole 'Author'
%  pdfsubject={Typ práce}, 						  	% Pole 'Subject'
%  pdfkeywords={Klíčová slova}           	% Pole 'Keywords'
%}
\hypersetup{pdfpagemode=FullScreen}       % otevření rovnou v režimu celé obrazovky
%%%%%%%%%%%%%%%%%%%%%%%%%%%%%%%%%%%%%%%%%%%%%%%%%%%%%%%%%%%%%%%%%%%%%%%

\usetheme{VUT} 				% barvy a rozložení prezentace odpovídající VUT FEKT
% alternativně lze použít jiná berevná témata, ale bez záruky. Například: 
%\usetheme{Darmstadt} \usecolortheme{default2}
\logoheader					% vytvoření zkráceného loga VUT FEKT v hlavičce slajdu, nechte odkomentované



\begin{document}

% v případě zakomentování následujícího se zobrazí v pravém dolním rohu slajdů klikatelné navigační symboly 
\disablenavigationsymbols

% titulní snímek, vysazen bez horních, dolních a postranních lišt (volba plain),
% není tak vysazen ani nadpis snímku
\maketitle

%%%%%%%%%%%%%%%%%%%%%%%%%%%%%%%%%%%%%%%%%%%%%%%%%%%%%%%%%%%%%%%%%%%%%%%
% 1. snímek s cíli (zadaním) práce
\begin{frame} 
	\frametitle{Základní informace}
	\begin{columns}[T] 								% prostředí sloupce s umístěním nahoře
		\begin{column}{0.5\textwidth}		% první sloupec
			Co je to Semafor?
			\begin{itemize}
				\item Outdoorová aplikace
				\item Edukační účely, táborové hry
				\item Zástupná funkce organizátora
				\item Dotykové senzory
				\item Světelné a zvukové signalizace
			\end{itemize}
		\end{column}
		%
		\begin{column}{0.5\textwidth}		% druhý sloupec
				Požadavky:
					\begin{itemize}
						\item Kompaktnost
						\item Bezpečnost
						\item Široké využití
						\item Voděodolnost
						\item Nízkou cenu
						\item Jednoduchost
						\item Možnost komunikace mezi jednotlivými Semafory
						\item Bezdrátová konfigurovatelnost
					\end{itemize}
				%\includegraphics[width=0.8\columnwidth]{obrazky/soucastky}
				%lze vložit popisek, ale povetšinou je to v prezentaci zbytečné
				%\caption{Popisek obrázku}%
				%\label{obr:ukazka}
		\end{column}
	\end{columns}	
\end{frame}

\begin{frame} 
	\frametitle{Cíle práce}

	\begin{columns}[T] 								% prostředí sloupce s umístěním nahoře
		\begin{column}{0.25\textwidth}		% první sloupec
			\begin{itemize}
				\item Nastudovat
				\item Porovnat
				\item Vybrat
				\item Navrhnout 
		\end{itemize}
		\end{column}
		%
		\begin{column}{0.65\textwidth}		% druhý sloupec
			\begin{figure}%	
				\centering
				\vspace{0.4cm}	              % horizontální mezera
				\includegraphics[width=1\columnwidth]{obrazky/zakladni_blokove_schema_prezentace.jpg}
				%\includegraphics[width=0.8\columnwidth]{obrazky/soucastky}
				%lze vložit popisek, ale povetšinou je to v prezentaci zbytečné
				%\caption{Popisek obrázku}%
				%\label{obr:ukazka}
			\end{figure}
		\end{column}
		\begin{column}{0.1\textwidth}		% druhý sloupec
		\end{column}
	\end{columns}	
		
	\end{frame}


%%%%%%%%%%%%%
\begin{frame} 
	\frametitle{Základní prvky}
	
	\begin{columns}[T] 								% prostředí sloupce s umístěním nahoře
		\begin{column}{0.5\textwidth}		% první sloupec
			%Obrázek znázorňuje model:\\[2ex]
			%
			Bezdrátová komunikace
			\begin{itemize}
				\item LoRa
				\begin{itemize}
					\item Dosah až 20 km 
					\item Bezlicenční pásmo
					\item Topologie hvězdy 
					\item Obousměrná komunikace
				\end{itemize}
				\item WiFi
				\begin{itemize}
					\item Bezlicenční pásmo
					\item Rozšířená 
					\item V každém mobilním telefonu
				\end{itemize}
			\end{itemize}
		\end{column}
		%
		\begin{column}{0.5\textwidth}		% druhý sloupec
			Mikrokontrolér
			\begin{itemize}
				\item ESP32-C3-MINI-1
				\item WiFi s anténou 
				\item SPI, UART, $I^2C$, USB
				\item 13 GPIO pinů
			\end{itemize}
		\end{column}
	\end{columns}											% ukončení prostředí sloupce
\end{frame}


\begin{frame} 
	\frametitle{Napájení}
	
	\begin{columns}[T] 								% prostředí sloupce s umístěním nahoře
		\begin{column}{0.4\textwidth}		% první sloupec
			%Obrázek znázorňuje model:\\[2ex]
			%
			\begin{itemize}
				\item Baterie LiFePO4
				\begin{itemize}
					\item 3 až 3,6 V
					\item Nejbezpečnější
					\item Životnost až 7000 cyklů 
					\item Teplotně stabilní
					\item Nehořlavé
					\item Netoxické
				\end{itemize}
				\item Nabíjecí obvod CN3058E
				\begin{itemize}
					\item Určený pro LiFePO4
					\item Nastavitelný nabíjecí proud
					\item Signalizace stavu nabíjení 
				\end{itemize}
				\item Měření napětí na baterii 
			\end{itemize}
		\end{column}
		%
		\begin{column}{0.6\textwidth}		% druhý sloupec
			\begin{figure}%	
				\centering
				\vspace{1cm}	              % horizontální mezera
				\includegraphics[width=1\columnwidth]{obrazky/CN3058E.png}
			\end{figure}
		\end{column}
	\end{columns}											% ukončení prostředí sloupce
\end{frame}

\begin{frame} 
	\frametitle{Komunikace s okolím}
	
	\begin{columns}[T] 								% prostředí sloupce s umístěním nahoře
		\begin{column}{0.5\textwidth}		% první sloupec
			%Obrázek znázorňuje model:\\[2ex]
			Světelná signalizace
			\begin{itemize}
				\item Programovatelné LED WS2812C
				\item Možnost spojení za sebou
				\item Převodník úrovní
				\item Zvyšovač napětí LT1930
			\end{itemize}
			\begin{figure}%	
				\centering	          
				\includegraphics[width=1\columnwidth]{obrazky/WS2812C_prezentace.png}
			\end{figure}
		\end{column}
		%
		\begin{column}{0.5\textwidth}		% druhý sloupec
			Senzory doteku
			\begin{itemize}
				\item Kapacitní tlačítka
				\item Mechanická odolnost
				\item Životnost
				\item Převodník
				\begin{itemize}
					\item Komunikace po  $I^2C$
					\item Až 7 tlačítek
				\end{itemize}
			\end{itemize}
			\begin{figure}%	
				\centering
				\vspace{1cm}	              % horizontální mezera
				%\includegraphics[width=0.8\columnwidth]{obrazky/soucastky}
				%lze vložit popisek, ale povetšinou je to v prezentaci zbytečné
				%\caption{Popisek obrázku}%
				%\label{obr:ukazka}
			\end{figure}
		\end{column}
	\end{columns}											% ukončení prostředí sloupce
\end{frame}

\begin{frame} 
	\frametitle{Budoucnost}
	\begin{itemize}
		\item Návrh DPS
		\item Výroba a oživení DPS
		\item Vytvoření her 
		\item Webová stránka pro konfiguraci
	\end{itemize}
\end{frame}


%%%%%%%%%%%%%
\begin{frame} 
	\frametitle{Závěr}

	\begin{center}
		\begin{columns}[T] 								% prostředí sloupce s umístěním nahoře
			\begin{column}{0.3\textwidth}		% první sloupec
				\begin{itemize}
					\item Bezdrátový modul
					\item Mikrokontrolér
					\item Napájení 
					\item Senzory doteku
					\item Signalizační prvky 
					\item Návrh kompletní elektroniky
				\end{itemize}
			\end{column}
			%
			\begin{column}{0.7\textwidth}		% druhý sloupec
				\begin{figure}
					\centering
					\vspace{0.5cm}
					\includegraphics[width=0.9\columnwidth]{obrazky/vysledne_blokove_schema_prezentace.jpg}
				\end{figure}
			\end{column}
		\end{columns}	
	\end{center}
	
\end{frame}


% podekovani
\begin{frame}[c] 
% bez nadpisu snímku
	\frametitle{\mbox{ }}
	\begin{center}
		\begin{columns}[T] 								% prostředí sloupce s umístěním nahoře
			\begin{column}{0.3\textwidth}		% první sloupec
				{\Huge Děkuji za pozornost!}
				\vspace{1cm}
				\begin{itemize}
					\item Bezdrátový modul
					\item Mikrokontrolér
					\item Napájení 
					\item Senzory doteku
					\item Signalizační prvky 
					\item Návrh kompletní elektroniky
				\end{itemize}
			\end{column}
			%
			\begin{column}{0.7\textwidth}		% druhý sloupec
				\begin{figure}
					\centering
					\vspace{0.5cm}
					\includegraphics[width=0.9\columnwidth]{obrazky/vysledne_blokove_schema_prezentace.jpg}
				\end{figure}
			\end{column}
		\end{columns}	
	\end{center}
\end{frame}

% otázky oponenta
%\frame{
%\frametitle{Budoucnost}
%	\emph{Jaká je souvislost Vašeho vzorce (1.2) s~Maxwellovými rovnicemi v~integrálním tvaru?}\\[2ex]
	%
%	Již staří Římané\,\dots
%}

\end{document}
