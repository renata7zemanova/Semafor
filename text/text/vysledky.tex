\chapter{Výběr a návrh elektroniky}
Výběr elektronických součástek probíhal dle jejich parametrů, využití, dostupnosti a ceny. Nejdříve byla vybrána bezdrátová technologie a mikrokontrolér a v závislosti 
na tom vše ostatní. 

\section{Bezdrátová technologie}
Ke komunikaci Semaforů mezi sebou byla zvolena technologie LoRa. Tato technologie byla zvolena především kvůli komunikačnímu dosahu. Jedná se sice o dražší technologii, 
ale na tolik, aby ji nebylo možné v tomto zařízení použít. Táborové hry se většinou hrají na loukách, které mají rozlohu několik stovek metrů čtverečných. LoRa je jedinou
dostupnou technologií, která na tyto vzdálenosti spolehlivě komunikuje. Bezdrátové propojení Semaforů bude použito pro posílání informací o aktuálně svítící barvě, či 
stisku tlačítka v závislosti na hře, která se aktuálně hraje. U některých her například může být žádoucí, aby po přepínání nesvítily všechny Semafory stejnou barvou 
a díky této komunikaci bude moci být takovým stavům zabráněno. 

%text o konkrétním vybraném LoRa modulu

K propojení Semaforu s telefonem a dalšími zařízeními byla vybrána technologie WiFi. Jedná se o rozšířenou technologii, která je v telefonech a noteboocích zabudovaná. 
Propojení bude tedy jednoduché a nastavovat hry se mohou na webové stránce, kde bude seznam her, které Semafor umí. U jednotlivých her se poté budou moci nastavovat 
další parametry. Po nastavení se konfigurace pošle do Semaforu. 

\section{Mikrokontrolér}
WiFi modul obsahuje jako jediný mikrokontrolér od firmy Espressif z řady ESP32. Konkrétně jde o typ ESP32-C3-MINI-1, dále již jen ESP32-C3. Je také nabízen za cenu, která 
je v porovnání s ostatními nízká a v porovnání s nabízenými parametry bezkonkurenční. Pro zařízení Semaforu je také se svým počtem periferií dostačující. ESP32-C3 má 
384 kB ROM a 4 MB flash paměti \cite{ESP_C3_dtsh}. Dále obsahuje WiFi modul pracující na frekvenci 2,4 GHz a Bluetooth \cite{ESP_C3_dtsh}. ESP32-C3 obsahuje mnoho periferií
jako je SPI, UART, $I^2C$, USB a další \cite{ESP_C3_dtsh}. Mikrokontrolér má vyvedeno 13 GPIO pinů, které je možno softwarově nastavit jako vstupní nebo výstupní. Tyto piny
slouží pro připojení senzorů, díky kterým je zprostředkována komunikace mezi mikrokontrolérem a okolním světem. V mikrokontroléru je také zabudován krystal s vlastní frekvencí 
40 MHz a v rámci pouzdra je také anténa pro WiFi \cite{ESP_C3_dtsh}.

Rozsah napájecího napětí je 3 až 3,6 V \cite{ESP_C3_dtsh}. Jeho maximální proudový odběr je 0,5 A \cite{ESP_C3_dtsh}. Mikrokontrolér ESP32-C3 garantuje pracovní teplotu 
od -45 °C až do 85 °C \cite{ESP_C3_dtsh}.

\begin{figure}[!h]
  \begin{center}
    \includegraphics[scale=0.8]{obrazky/blokove_schema_MCU.jpg}
  \end{center}
  \caption[Blokové schéma mikrokontroléru ESP32-C3]{Blokové schéma mikrokontroléru ESP32-C3 \cite{ESP_C3_dtsh}.}
\end{figure}

K pinu 3V3, který slouží pro připojení napájecího napětí jsou také připojeny kondenzátory o hodnotě 10 $\mu$F a 100 nF dle doporučení z dokumentace \cite{ESP_C3_dtsh}. Tyto 
kondenzátory slouží pro filtraci napájecího napětí, aby bylo vyfiltrováno případné rušení o různých frekvencích.

Pin EN slouží pro povolení funkce mikrokontroléru. Tento pin nesmí zůstat nezapojený, tzv. floating. Jeho zapojení je převzato z dokumentace, tj. pullup rezistor o hodnotě 
10 k$\Omega$ a ke GND je připojen přes kondenzátor o hodnotě 1 $\mu$F \cite{ESP_C3_dtsh}.

\begin{figure}[!h]
  \begin{center}
    \includegraphics[scale=0.5]{obrazky/ESP32-C3.png}
  \end{center}
  \caption[Schéma zapojení mikrokontroléru ESP32-C3]{Schéma zapojení mikrokontroléru ESP32-C3.}
\end{figure}

ESP32-C3 má konfigurační piny, které slouží při restartu pro určení, odkud bude načten program pro mikrokontrolér. Tyto piny musí být při restartu v daném nastavení. 
Konfiguračními piny jsou GPIO2, GPIO8 a GPIO9. Piny GPIO8 a GPIO9 nesmí být nikdy nastaveny současně do logické nuly. 

\begin{table}[!h]
  \caption[Konfigurační piny ESP32-C3]{Konfigurační piny ESP32-C3 \cite{ESP_C3_dtsh}}
  \begin{center}
  	\small
	  \begin{tabular}{|c|c|c|c|}
	    \hline
	    \textbf{Pin}	& \textbf{Výchozí}	& \textbf{Načtení programu} & \textbf{Načtení programu} \\
      \textbf{}	& \textbf{}	& \textbf{z flash paměti} & \textbf{z bootloaderu} \\
	    \hline
	    \textbf{GPIO2}	& Není dostupný & 1 & 1 \\ 
	    \hline
	    \textbf{GPIO8}	& Není dostupný & Nezáleží & 1 \\ 
	    \hline
	    \textbf{GPIO9} & Interní měkký pullup & 1 & 0 \\
	    \hline
	  \end{tabular}
  \end{center}
\end{table}

Protože bude využito programování přes USB piny D+ a D-, tak není zapotřebí načtení z bootloaderu a je tedy zapotřebí všechny konfigurační piny při restartu připojit do 
logické jedničky. Do logické jedničky lze připojit přes pullup rezistor. Pullup rezistory má sběrnice $I^2C$, takže na GPIO8 a GPIO9 byla připojena právě sběrnice $I^2C$.
Pin GPIO2 nebyl využit pro připojení žádného senzoru, a proto byl připojen přes pullup rezistor o hodnotě 10 k$\Omega$ k napájecímu napětí.

\section{Napájení}
Zabudování baterie přináší kompaktnost řešení a pro použití není třeba dalších komponent. Pokud je ale na táboře větší využití, tak se baterie vybije.
Na táborech většinou nebývá k dispozici připojení k elektrické síti, a proto je řešením powerbanka. Na Semaforu tedy bude napájecí vstup USB-A pro nabíjení baterií
přímo z powerbanky bez nutnosti kabelu. Semafor musí být koncipován tak, aby se mohla baterie nabíjet a zároveň, aby při tom byly Semafory funkční.

Při realizaci Semaforu byla tedy zvolena kombinace napájení pomocí baterií i pomocí powerbanky. Článek baterie LiFePO4 byl vybrán právě kvůli již zmíněným vynikajícím 
vlastnostem. Vybraný mikrokontrolér má napájecí napětí v rozsahu 3 až 3,6 V \cite{ESP_C3_dtsh}. Pro funkci mikrokontroléru tedy nebude muset být použit ani 
převodník napětí. 

Napětí na baterii je měřeno pomocí děliče a připojeno na pin GPIO4, který má k dispozici AD převodník \cite{ESP_C3_dtsh}. V softwaru bude nastaven útlum na 0 dB, to 
odpovídá rozsahu měřeného napětí od 0 do 700 mV \cite{ESP_C3_tech_ref}.
Proto je napětí z baterie pomocí rezistorového děliče převedeno na rozsah od 0 do 600 mV. Zde je pomocí AD převodníku napětí na baterii měřeno. Spodní rezistor děliče byl
zvolen o hodnotě 10 k$\Omega$ a druhý byl dopočítán na hodnotu 50 k$\Omega$ podle maximálního napětí baterií LiFePO4 3,6 V. Nejbližší hodnota rezistoru je 47 k$\Omega$
\cite{rezistorova_rada}. Tomu odpovídá maximální napětí na AD převodníku 0,63 V, což je stále v možném měřeném rozsahu. 

Při nízkém napětí baterie jsou softwarově odpojovány periferie a senzory od jejich napájecího napětí. Děje se tak pomocí pinu GPIO5, na který je připojen spínací 
tranzistor.

\subsection{Nabíjecí obvod}
Nabíjecí obvody jsou závislé na konkrétním typu baterií, které budou nabíjeny. Vzhledem k vybranému typu baterií LiFePO4 byly uvažovány pouze komerčně
dostupné integrované obvody, které jsou určeny pro nabíjení tohoto typu baterií. 

Vybraný typ baterií LiFePO4 lze nabíjet pomocí obvodu CN3058E. 

Nabíjecí obvod CN3058E je určen pro nabíjení pouze LiFePO4 baterií a lze jím napájet právě 1 článek těchto baterií \cite{charger_dtsh}. Napájecí napětí tohoto 
nabíjecího čipu se pohybuje mezi 3,8 až 6 V \cite{charger_dtsh}. Díky tomu lze přímo použít napětí z USB konektoru. 

Když je nabíjecí obvod odpojen od napájecího napětí, tak přejde do režimu spánku \cite{charger_dtsh}. V tomto režimu je baterie vybíjena proudem menším než 
3 $\mu$A \cite{charger_dtsh}. Tento proud je oproti klidovým proudům jiných součástek zanedbatelný, a proto nemusí být baterie od nabíjecího obvodu odpojována,
když není nabíjena. 

Nabíjecí obvod CN3058E umí také vyhodnocovat teplotu baterie a v závislosti na tom přestávat baterii nabíjet \cite{charger_dtsh}. Tato funkce není v zapojení
Semaforu využita, proto je pin TEMP připojen k signálu GND \cite{charger_dtsh}.

Tento nabíjecí obvod se vyrábí ve standardizovaném pouzdře SOP8 \cite{charger_dtsh}.

\subsection{Zapojení nabíjecího obvodu}
Rezistor připojený k pinu ISET slouží pro nastavení hodnoty nabíjecího proudu \cite{charger_dtsh}. V tomto zapojení byl počítán pro nabíjecí proud 1 A dle rovnice
z dokumentace: 
\begin{equation} 
  R_{8}~=~\frac{1218}{I_{CH}}~=~\frac{1218}{1}~=~1,218~k\Omega. 
  \quad \quad \quad \quad \quad \quad \quad \quad \quad \cite{charger_dtsh}
\label{eq:I_CH}
\end{equation}

Z výpočtu vyplývá, že rezistor by měl mít hodnotu 1,218 k$\Omega$. Nejbližší hodnota z rezistorové řady E12 je hodnota 1,2 k$\Omega$, proto byl také zvolen rezistor 
o této hodnotě \cite{rezistorova_rada}. Odpovídá tomu nabíjecí proud 1015 mA, který nebude mít vliv na životnost baterií. 

Vstupní a výstupní kondenzátory slouží pro filtraci zákmitů napájecího napětí a také napětí, kterým je nabíjena baterie. Hodnoty kondenzátorů byly převzaty
z doporučení z dokumentace, tj. 4,7 $\mu$F \cite{charger_dtsh}.

Kladný pól nabíjené baterie je připojen na pinu BAT, záporný pól je připojen ke GND. Pin BAT poskytuje nabíjecí proud do baterie a zároveň poskytuje konstantní 
nabíjecí napětí. V režimu spánku je svodový proud tohoto pinu 3 $\mu$A \cite{charger_dtsh}. 

Pin VIN slouží pro napájení vnitřního obvodu CN3058E. Je na něj přikládáno napájecí napětí z USB, tedy 5 V. Pokud napájecí napětí klesne na napětí o 10 mV nižší, 
než je napětí na pinu BAT, tak vnitřní obvod přechází do režimu spánku \cite{charger_dtsh}. V tomto režimu klesá proud pinu BAT na méně než 3 $\mu$A \cite{charger_dtsh}.

Tento nabíjecí obvod má možnost indikace nabíjení baterií a dokončení nabíjení. Tato indikace je realizována pomocí 2 LED připojených přes pullup rezistor. Hodnota
pullup rezistoru byla převzata z doporučení z dokumentace, tj. 330 $\Omega$. Červená LED indikuje nabíjení baterií a je připojena na pin /CHRG a zelená LED indikuje 
dokončené nabíjení a je připojena na pin /DONE. Obě LED jsou k pinům nabíjecího čipu připojeny katodou. 

Obvod CN3058E může také měřit teplotu na nabíjené baterii. Slouží k tomu vstupní pin TEMP. Měření probíhá pomocí odporového děliče, jehož střed je připojen na snímač 
teploty. Tento snímač je připojen na baterii. Pokud je napětí na pinu TEMP nižší než 45 \% nebo vyšší než 80 \% úrovně napájecího napětí, tak je indikována moc nízká
nebo moc vysoká teplota baterie a nabíjení je zastaveno \cite{charger_dtsh}. Jinak nabíjení pokračuje. Uzemněním pinu TEMP je funkce měření teploty deaktivována 
\cite{charger_dtsh}. V této práci není měření teploty baterií využíváno, a proto je pin TEMP připojen ke GND. 

\begin{figure}[!h]
  \begin{center}
    \includegraphics[scale=0.6]{obrazky/CN3058E.png}
  \end{center}
  \caption[Schéma zapojení nabíjecího obvodu pro LiFePO4]{Schéma zapojení nabíjecího obvodu pro LiFePO4.}
\end{figure}

\section{Senzory doteku}
V návrhu Semaforu byla zvolena kapacitní dotyková tlačítka. Pro možnost použití uvnitř i venku jsou díky možnosti voděodolnosti 
vhodnějším řešením. Také velikost a označení tlačítka může být variabilní. Velikost může být na DPS navržena dle potřeby a potisk
v místě tlačítka vyznačen barevně, nebo např. samolepkou. Odezva na dotyk bude realizována pomocí vibračního motoru.

\section{Vibrační motor}
Vibrační motory jsou založeny na principu kmitání. Motor je připevněn k zařízení, které je kmitáním rozvibrováno. Vibrační motory jsou dnes 
nedílnou součástí mnoha elektronických zařízení včetně mobilního telefonu nebo dětských hraček. 

Dioda slouží jako ochrana proti přepětí, protože motor je indukční zátěž, takže vytváří napěťové špičky. Díky diodě je mikrokontrolér chráněn 
proti špičkovému napětí, které by se na něj mohlo dostat. Kondenzátor slouží k tomu, aby napěťové špičky eliminoval, nebo alespoň zmenšoval. 

Vibrační motor je připojen k mikrokontroléru přes tranzistor, protože maximální výstupní proud z pinu MCU není dostatečně velký na to, aby 
motor roztočil. Tranzistor je tedy připojen na gate tranzistoru, který se při logické jedničce na pinu sepne a motorem protéká proud, který 
nedodává MCU, ale zdroj 3.3 V (v tomto případě baterie LiFePO4). Baterie tak dokáže dodat dostatek proudu, aby se motor roztočil. 

Pro Semafor byl vybrán vibrační motor LCM1020A2945F. Tento motor má maximální požadovaný proud 120 mA \cite{vib_motor_dtsh}. Maximální proud, 
který lze odebírat z pinu mikrokontroléru ESP32-C3, je 40 mA \cite{ESP_C3_dtsh}. Vibrační motor lze pouze spínat, nebo je možné jej připojit 
k pinu, který dokáže generovat PWM a lze tím regulovat jeho otáčky. 

Vibrační motor slouží jako odezva na dotyk kapacitního tlačítka. 

\begin{figure}[!h]
  \begin{center}
    \includegraphics[scale=0.4]{obrazky/Vibracni_motor.png}
  \end{center}
  \caption[Schéma zapojení vibračního motoru]{Schéma zapojení vibračního motoru.}
\end{figure}

%fotka vybraného motoru

\section{Převodník pro kapacitní tlačítka}
Mikrokontrolér ESP32-C3 nemá kapacitní vstupy, proto je zapotřebí kapacitní dotyková tlačítka připojit přes převodník. Je zapotřebí připojit 
5 tlačítek. 

Vybraný převodník AT42QT1070 dokáže pracovat ve 2 režimech. V prvním režimu může být zapojeno maximálně 5 kapacitních tlačítek, která jsou připojena
k pinům KEY0 až KEY4. Jako výstup se používají piny OUT0 až OUT4. Každé tlačítko má tedy svůj výstup, který může být připojen k GPIO pinům MCU 
nebo k nim mohou být připojeny např. LED \cite{conv_cap_but_AT42QT1070_dtsh}. 

Druhý režim je využitelný pouze v případě, je-li převodník připojen k MCU. V tomto případě může být k převodníku připojeno až 7 kapacitních tlačítek, 
která jsou připojena na pinech KEY0 až KEY6. Převodník poté komunikuje s MCU pomocí komunikační sběrnice $I^2C$ \cite{conv_cap_but_AT42QT1070_dtsh}. 
Z registru převodníku lze poté vyčíst stavy daných kapacitních dotykových tlačítek. 

Jelikož je v tomto návrhu Semaforu využit mikrokontrolér, který podporuje komunikaci po sběrnici $I^2C$, tak bylo využito zapojení právě s tímto typem 
komunikace. Díky tomu budou využity pouze 2 GPIO piny mikrokontroléru ESP32-C3 a ne 5 GPIO pinů, které by byly zapotřebí při zapojení bez komunikace pro
sběrnici $I^2C$.

Převodník má kondenzátory C3 a C4 připojeny na napájecím pinu vůči GND, aby nebyly případné proudové špičky přivedeny na napájení převodníku. Rezistory
R17 a R18 slouží jako pullup rezistory při komunikaci pomocí sběrnice $I^2C$ s mikrokontrolérem EP32-C3. Na piny KEY0 až KEY4 jsou připojena kapacitní 
dotyková tlačítka.  

Pin MODE je připojen k signálu GND, protože převodník je provozován v režimu komunikujícím přes $I^2C$ sběrnici \cite{conv_cap_but_AT42QT1070_dtsh}.

Pin /CHANGE je připojen k GPIO pinu mikrokontroléru. Slouží pro indikaci změny stavu některého z připojených tlačítek \cite{conv_cap_but_AT42QT1070_dtsh}. 
Signál z tohoto pinu lze tedy využít jako indikátor vyvolání přerušení pro obsluhu tlačítek. 

\begin{figure}[!h]
    \begin{center}
      \includegraphics[scale=0.4]{obrazky/AT42QT1070.png}
    \end{center}
    \caption[Zapojení převodníku AT42QT1070 pro kapacitní tlačítka]{Zapojení převodníku AT42QT1070 pro kapacitní tlačítka.}
\end{figure}

\section{Světelná signalizace}
Pro realizaci signalizace přítomnosti napájecího napětí byly vybrány neprogramovatelné LED. Zelené LED byly vybrány dvě, jedna pro 
indikaci přítomnosti napětí 5 V a druhá pro indikaci přítomnosti napětí 3,3 V z baterií. Tyto LED budou použity pouze na prototypu 
pro ulehčení oživování. Dále budou odstraněny kvůli šetření energie, protože jde o bateriově napájené zařízení. Přítomnost svitu 
tzv. power LED by také mohla mást při hře nebo různých úkolech.

Pro realizaci světelné signalizace pro hry byly vybrány inteligentní programovatelné LED typu WS2812C. Bylo jich použito 12, protože 
z dvanácti LED lze jednoduše zhotovit ciferník pro odpočítávání času a také je lze rozdělit na segmenty na třetiny nebo čtvrtiny. 

Komunikační napěťová úroveň logické jedničky těchto LED by měla být alespoň na úrovni 70 \% napájecího napětí \cite{WS2812C_dtsh}. 
Protože použitý mikrokontrolér ESP32-C3 má komunikační napěťovou úroveň logické jedničky jeho napájecí napětí, což je 3 až 3,6 V, 
tak je zapotřebí využít převodník napěťové úrovně \cite{ESP_C3_dtsh}. Komunikace je v tomto případě pouze jednosměrná, 
to znamená, že MCU posílá data do LED, ale LED neposílají žádná data do MCU. Převodník je realizován unipolárním tranzistorem 
a jedním pullup rezistorem. Rezistor je připojen k napájecímu napětí inteligentních LED WS2812C. 
Tranzistor Q1 má gate připojený k napájecímu napětí MCU. Pokud bude mikrokontrolér do LED posílat logickou jedničku, tak bude rozdíl
mezi gate a source 0 V. Tím pádem bude tranzistor uzavřený a tím se přes rezistor R4 připojí k LED jejich napájecí napětí. Toto napětí 
je pro inteligentní LED logickou jedničkou. Pokud bude MCU posílat logickou nulu, tedy 0 V, tak je rozdíl napětí mezi gate a source 
napájecí napětí mikrokontroléru. Tranzistor je tedy otevřený a tím se napětí 0 V dostane k inteligentním LED a na rezistoru se objeví
úbytek napětí o velikosti napájecího napětí inteligentních LED. Napětí 0 V je logickou nulou i pro inteligentní LED. Tento převodník
je určen pouze pro komunikaci jedním směrem. 

\begin{figure}[!h]
  \begin{center}
    \includegraphics[scale=0.6]{obrazky/prevodnik_urovni_pro_WS2812C.png}
  \end{center}
  \caption[Zapojení převodníku úrovní pro WS2812C]{Zapojení převodníku úrovní pro WS2812C.}
\end{figure} 

\begin{figure}[!h]
  \begin{center}
    \includegraphics[scale=0.5]{obrazky/WS2812C.png}
  \end{center}
  \caption[Zapojení inteligentních LED WS2812C]{Zapojení inteligentních LED WS2812C.}
\end{figure}

Kondenzátor u každé LED slouží pro filtraci napájecího napětí. 

Tyto programovatelné LED mají maximální spotřebu 5 mA na jeden kanál. Při zapnutí všech kanálů (svícení bílou) je maximální
spotřeba jedné LED 15 mA \cite{WS2812C_dtsh}. Pokud LED nesvítí, tak je její maximální klidový proud 0,3 mA \cite{WS2812C_dtsh}.
Při použití 12 LED je tedy maximální odběr všech LED 180 mA.

Pro napájení těchto inteligentních LED je zapotřebí napětí v rozsahu 4,5 až 5,5 V \cite{WS2812C_dtsh}. 
Použité baterie LiFePO4 mají napětí pouze 3,2 V, proto je zapotřebí použít zvyšovač napětí na 5 V. 

\subsection{Zvyšovač napětí pro programovatelné LED}
Z komerčně dostupných integrovaných obvodů byl hledán zvyšovač napětí, který vytváří z napětí 3,3 V napětí 5 V a může přitom dodávat do výstupu proud alespoň 200 mA. 
Maximální odběr všech dvanácti potřebných inteligentní LED má maximální odběr 180 mA. S rezervou je tedy zapotřebí proud alespoň 200 mA. Nalezené obvody, které vyhovují 
těmto parametrům jsou LT1930 a MCP1640. 

Obvod LT1930 v doporučeném zapojení při vstupním napětí 3,3V vytváří výstupní napětí o hodnotě 5 V s maximálním odběrem proudu 480 mA \cite{LT1930_dtsh}. Napájecí napětí 
tohoto obvodu je v rozsahu 2,45 V až 16 V, což vyhovuje napájecímu napětí z baterií LiFePO4 \cite{LT1930_dtsh}.

Obvod MCP1640 v doporučeném zapojení s rozsahem vstupního napětí 3 až 4,2 V vytváří výstupní napětí o hodnotě 5 V s maximálním odběrem proudu 300 mA \cite{MCP1640_dtsh}.

Byl vybrán zvyšovač napětí LT1930, díky své lepší dostupnosti v této době nedostatku čipů, a také dokáže do výstupu dodat vyšší proud. Zapojení obou čipů je téměř totožné. 

Pin /SHDN slouží k zapínání a vypínání obvodu. Pomocí přiloženého napětí 2,4 V a více na tento pin je obvod zapnut \cite{LT1930_dtsh}. Pin SW slouží pro  připojení cívky, 
případně diody, aby se snížilo elektromagnetické rušení \cite{LT1930_dtsh}. 

Shottkyho dioda byla vybrána dle doporučení z dokumentace. Byla vybrána dioda typu MBR0520, protože maximální napětí na diodě nepřekročí 20 V a protékající proud nepřesáhne 
0,5 A \cite{LT1930_dtsh}.

Byla vybrána cívka, která odpovídá doporučení z dokumentace. Přesný typ, který byl v dokumentaci zmíněn nebyl k dispozici, a proto byl vybrán typ velmi podobný a vlastnostmi 
srovnatelný. Cívka CDRH3D18NP-4R7NC má feritové jádro, které je pro funkci požadováno \cite{LT1930_dtsh}. Pro typ LT1930 by měl být proud, který cívkou může protékat, alespoň
1A a její indukčnost by měla být 4,7 $\mu$H nebo 10 $\mu$H \cite{LT1930_dtsh}. Vybraná cívka má indukčnost 4,7 $\mu$H, proud, který jí může protékat, je 1,35 A a její rozměry 
jsou 3,8 $\times$ 3,8 $\times$ 2 mm \cite{civka_dtsh}.

Pin FB slouží  pro zapojení zpětné vazby napětí na baterii. Jeho referenční napětí musí být nastaveno v rozmezí 1,240 V až 1,270 V, typická hodnota je však 1,255 V \cite{LT1930_dtsh}. 
Pro výstupní napětí 5 V byl zvolen rezistor R10 o hodnotě 13 k$\Omega$ z rezistorové řady E24 \cite{rezistorova_rada}. Řada E24 byla zvolena kvůli požadované přesnosti
napětí na pinu FB obvodu LT1930. Napětí na rezistoru R10 musí být tedy 1.255 V. Na rezistoru R9 je tedy úbytek napětí 3,745 V. Pomocí trojčlenky byla dopočítána hodnota 
rezistoru R9 dle rovnice:
\begin{equation} 
  R_{9}~=~\frac{R_{10}~\cdot~U_{R9}}{U_{R10}}~=~\frac{13~\cdot~3,745}{1,255}~=~38,79~\:k\Omega. 
  \quad
\label{eq:R9}
\end{equation}

Nejbližší hodnota rezistoru z rezistorové řady E24 je 39 k$\Omega$ \cite{rezistorova_rada}. Reálná hodnota napětí na rezistoru R10, tj. napětí na pinu FB byla dopočítána
dle rovnice:
\begin{equation} 
  U_{R10}~=~\frac{U_{OUT}}{R_{9}~+~R_{10}}~\cdot~R_{10}~=~\frac{5}{39~+~13}~\cdot~13~=~1,25~V. 
  \quad
\label{eq:UR10}
\end{equation}

Napětí 1,25 V je v povoleném rozmezí napětí na pinu FB. 

Přesné výstupní napětí se spočítá podle vzorce:
\begin{equation} 
  U_{OUT}~=~U_{FB}~\cdot~(1~+~\frac{R_{9}}{R_{10}})~=~1,25~\cdot~(1~+~\frac{39}{13})~=~5~V. 
  \quad \quad \quad \quad \cite{LT1930_dtsh}
\label{eq:VOUT_LT1930}
\end{equation}

\begin{figure}[!h]
  \begin{center}
    \includegraphics[scale=0.7]{obrazky/LT1930.png}
  \end{center}
  \caption[Zapojení zvyšovače napětí LT1930]{Zapojení zvyšovače napětí LT1930.}
\end{figure}

\section{Zvuková signalizace}
Zvuková signalizace může sloužit například pro potvrzení správnosti hesla, možnosti odejít na další stanoviště, vypršení času pro daný úkol a mnoho dalších. 

Jako zvuková signalizace bylo vybráno piezo s vlastním oscilátorem typu \\BMT1205XH7.5 \cite{piezo_dtsh}. Maximální odebíraný proud vybraného pieza je 30 mA a rezonanční frekvence 
je 2,3 kHz \cite{piezo_dtsh}. Intenzita zvuku pieza je ve vzdálenosti 10 cm od něj minimálně 83 dB \cite{piezo_dtsh}.

\begin{figure}[!h]
  \begin{center}
    \includegraphics[scale=0.45]{obrazky/piezo.png}
  \end{center}
  \caption[Schéma zapojení pieza]{Schéma zapojení pieza.}
\end{figure}

\section{Konektor}
Jako programovací konektor byl zvolen konektor USB-C. Tento konektor je v dnešní době velmi rozšířený a jeho použití se v následující době stále rozšiřuje. 

Konektoru USB-C je využíván pouze jako standardní a dostupný konektor, který je mezi běžnou
populací rozšířený a v následujících letech se bude rozšiřovat stále více. Je využito standardního jmenovitého napětí 5 V pro nabíjení baterií a nadále pinů D+ a D-, 
které jsou využity pro komunikaci při programování. 

Konektor USB-C je robustní a oboustranný, díky čemuž nebude docházet k tak častému poškození, jak by mohlo být např. u konektoru Micro USB. Při používání běžnou veřejností
se jedná o vítaný bonus. 

Vybraný mikrokontrolér ESP32-C3 umožňuje komunikaci přímo po USB protokolu a není díky tomu zapotřebí žádného převodníku pro komunikaci \cite{ESP_C3_dtsh}.

\begin{figure}[!h]
  \begin{center}
    \includegraphics[scale=0.6]{obrazky/USB_C.png}
  \end{center}
  \caption[Zapojení konektoru USB-C]{Zapojení konektoru USB-C.}
\end{figure}

\begin{figure}[!h]
  \begin{center}
    \includegraphics[scale=0.7]{obrazky/USB_A.png}
  \end{center}
  \caption[Zapojení konektoru USB-A]{Zapojení konektoru USB-A.}
\end{figure}

Připojené Shottkyho diody k napájecímu napětí slouží pro zadržení případného zpětného proudu. Shottkyho diody jsou dimenzovány na proud, který odebírá celé zařízení. Vybrané 
Shottkyho diody  B5819W mají maximální napětí 20 V, jmenovitý proud 1 A a maximální špičkový proud 9 A \cite{shotky_dtsh}.

Transily připojené k napájecímu pinu a ke komunikačním pinům D+ a D- slouží k ochraně proti přepětí a elektrostatickým výbojům o velikosti až 30 kV. 

Pro napájení pomocí powerbanky bez potřeby kabelu slouží konektor USB-A. 

Rezistory o hodnotě 5,1 k$\Omega$ na pinech CC1 a CC2 slouží pro signalizaci, že je k USB-C připojeno zařízení. Dle standardu USB-C totiž nabíječka bez 
připojení těchto rezistorů nesmí připojit napájecí napětí 5 V na pin VBUS \cite{USB-C}. 

%ke každé komponentě napsat odběr

\section{Výsledné zapojení}
Pro realizaci Semaforu byly vybrány následující komponenty: mikrokontrolér ESP32-C3, baterie LiFePO4, nabíjecí obvod CN3056E, konektory USB-C a USB-A,
kapacitní tlačítka s převodníkem AT42QT1070, inteligentní LED WS2812C s převodníkem napětí LT1930, piezo, vibrační motor a LoRa modul. LoRa modul komunikuje 
s mikrokontrolérem pomocí sběrnice UART, převodník pro kapacitní tlačítka pomocí sběrnice $I^2C$ a programování bude probíhat pomocí USB sběrnice. Vše je 
zapojeno dle následujícího blokového schématu. 

\begin{figure}[!h]
  \begin{center}
    \includegraphics[scale=0.65]{obrazky/vysledne_blokove_schema.jpg}
  \end{center}
  \caption[Výsledné blokové schéma Semaforu]{Výsledné blokové schéma Semaforu.}
\end{figure}




%\chapter{Návrh DPS}
%kulatá DPS - proč?

%\section{Kapacitní tlačítka} 
%Byl požadavek na 5 tlačítek. Jedno tlačítko je uprostřed a slouží jako hlavní tlačítko. U her bude používáno např. jako registrace průchodu místem apod. Bude tedy nejčastěji
%používáno a zároveň může být stisknuto, když hráč běží, takže by mělo být co nejjednodušeji stisknutelné. Proto bylo navrženo větší než zbylá tlačítka. Konkrétně má 
%5~$\times$~5~cm. Ostatní tlačítka slouží například jako směrovky, nebo pro vyklikávání nějakého kódu, aby získali nějakou informaci. Slouží tedy primárně, když účastník 
%u sebaforu stojí, nebo sedí, a vyklikává. Díky tomu mohou být tlačítka menší než hlavní tlačítko, konkrétně mají 2~$\times$~2~cm. Tato tlačítka jsou proto umístěna 
%po stranách hlavního tlačítka a jsou popsána BTN\_ENTER, BTN\_UP, BTN\_DOWN, BTN\_RIGHT a BTN\_LEFT.

%\section{LED}
%do sekce o DPS se zmínit o jejich umístění - kruh, hodiny, proto 12 ks. (lze také rozdělit na 3 segmenty)

%\section{Konektory}
%konektor USB C
%konektor USB A
%na kraji DPS