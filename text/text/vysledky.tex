\chapter{Návrh elektroniky}






\chapter{Návrh DPS}
%kulatá DPS - proč?

\section{Kapacitní tlačítka} 
Byl požadavek na 5 tlačítek. Jedno tlačítko je uprostřed a slouží jako hlavní tlačítko. U her bude používáno např. jako registrace průchodu místem apod. Bude tedy nejčastěji
používáno a zároveň může být stisknuto, když hráč běží, takže by mělo být co nejjednodušeji stisknutelné. Proto bylo navrženo větší než zbylá tlačítka. Konkrétně má 
5~$\times$~5~cm. Ostatní tlačítka slouží například jako směrovky, nebo pro vyklikávání nějakého kódu, aby získali nějakou informaci. Slouží tedy primárně, když účastník 
u sebaforu stojí, nebo sedí, a vyklikává. Díky tomu mohou být tlačítka menší než hlavní tlačítko, konkrétně mají 2~$\times$~2~cm. Tato tlačítka jsou proto umístěna 
po stranách hlavního tlačítka a jsou popsána BTN\_ENTER, BTN\_UP, BTN\_DOWN, BTN\_RIGHT a BTN\_LEFT.

\section{Konektory}
%konektor USB C
%konektor USB A