% Pro sazbu seznamu literatury použijte jednu z následujících možností

%%%%%%%%%%%%%%%%%%%%%%%%%%%%%%%%%%%%%%%%%%%%%%%%%%%%%%%%%%%%%%%%%%%%%%%%%
%1) Seznam citací definovaný přímo pomocí prostředí literatura / thebibliography

\begin{thebibliography}{99}
	
\bibitem{sr02/2009}
		VUT v~Brně:
    \emph{Úprava, odevzdávání a zveřejňování vysokoškolských kva\-li\-fi\-kač\-ních prací na VUT v~Brně}\/ [online].
		Směrnice rektora č.\,2/2009.
		Brno: 2009, po\-sled\-ní aktualizace 24.\,3.\,2009 [cit.\,23.\,10.\,2015].
    Dostupné z~URL:\\
    <\url{https://www.vutbr.cz/uredni-deska/vnitrni-predpisy-a-dokumenty/smernice-rektora-f34920/}>.

\bibitem{CSN_ISO_690-2011}
    \emph{ČSN ISO 690 (01 0197) Informace a dokumentace -- Pravidla pro bibliografické odkazy a citace informačních zdrojů.}
    40 stran. Praha: Český normalizační institut, 2011.

\bibitem{CSN_ISO_7144-1997}
    \emph{ČSN ISO 7144 (010161) Dokumentace -- Formální úprava disertací a podobných dokumentů.}
    24 stran. Praha: Český normalizační institut, 1997.

\bibitem{CSN_ISO_31-11}
    \emph{ČSN ISO 31-11 Veličiny a jednotky -- část 11: Matematické znaky a značky používané ve fyzikálních vědách a v~technice.}
    Praha: Český normalizační institut, 1999.

\bibitem{BiernatovaSkupa2011:CSNISO690komentar}
    BIERNÁTOVÁ, O., SKŮPA, J.:
    \emph{Bibliografické odkazy a citace dokumentů dle ČSN ISO 690 (01 0197) platné od 1.\,dubna 2011}\/ [online].
    2011, poslední aktualizace 2.\,9.\,2011 [cit. 19.\,10.\,2011].
    Dostupné z~URL:
    \(<\)\url{http://www.citace.com/CSN-ISO-690.pdf}\(>\)
%    \(<\)\href{http://www.boldis.cz/citace/citace.html}{http://www.boldis.cz/citace/citace.html}\(>\).

\bibitem{pravidla}
    \emph{Pravidla českého pravopisu}.
    Zpracoval kolektiv autorů. 1.\ vydání.
    Olomouc: FIN PUB\-LISH\-ING, 1998. 575 s. ISBN 80-86002-40-3.

\bibitem{Walter1999}
	WALTER, G.\,G.; SHEN, X.
	\emph{Wavelets and Other Orthogonal Systems}.
	2. vyd. Boca Raton: Chapman\,\&\,Hall/CRC, 2000. 392~s. ISBN 1-58488-227-1

\bibitem{Svacina1999IEEE}
	SVAČINA, J.
	Dispersion Characteristics of Multilayered Slotlines -- a Simple Approach.
	\emph{IEEE Transactions on Microwave Theory and Techniques},
	1999, vol.\,47, no.\,9, s.\,1826--1829. ISSN 0018-9480.

\bibitem{RajmicSysel2002}
    RAJMIC, P.; SYSEL, P.
    Wavelet Spectrum Thresholding Rules.
    In \emph{Proceedings of the International Conference Research in Telecommunication Technology},
    Žilina: Žilina University, 2002. s.\,60--63. ISBN 80-7100-991-1.

%moje zdroje
\bibitem{PrincipKapTl}
    VOJÁČEK, A.:
    \emph{Pravidla pro konstrukci kapacitních dotykových tlačítek mTouch}\/ [online].
    2008, poslední aktualizace 13.\,12.\,2008 [cit. 26.\,10.\,2022].
    Dostupné z~URL:
    \(<\)\url{https://automatizace.hw.cz/pravidla-pro-konstrukci-kapacitnich-dotykovych-tlacitek-mtouch}\(>\)

\bibitem{ESP_C3_dtsh}
    Espressif Systems:
    \emph{ESP32-C3-MINI-1}\/ [online].
    2022, poslední aktualizace 2022 [cit. 31.\,10.\,2022].
    Dostupné z~URL:
    \(<\)\url{https://www.espressif.com/sites/default/files/documentation/esp32-c3-mini-1_datasheet_en.pdf}\(>\)

\bibitem{charger_dtsh}
    CONSONANCE:
    \emph{1A LiFePO4 Battery Charger CN3058E}\/ [online].
    2022, poslední aktualizace 2022 [cit. 31.\,10.\,2022].
    Dostupné z~URL: %upravit údaje o datech
    \(<\)\url{http://www.consonance-elec.com/en/static/upload/file/20220425/1650867856106004.pdf}\(>\)

\bibitem{rezistorova_rada}
    Radioklub OK1KVK:
    \emph{Elektrotechnické řady hodnot E3, E6, E12, E24}\/ [online].
    2011, poslední aktualizace 25.\,05.\,2011 [cit. 31.\,10.\,2022].
    Dostupné z~URL: 
    \(<\)\url{https://ok1kvk.cz/clanek/2011/elektrotechnicke-rady-hodnot-e3-e6-e12-e24/}\(>\)

\bibitem{LT1930_dtsh}
    LINEAR TECHNOLOGY:
    \emph{LT1930/LT1930A}\/ [online].
    2001, poslední aktualizace 2001 [cit. 5.\,11.\,2022].
    Dostupné z~URL: 
    \(<\)\url{https://www.analog.com/media/en/technical-documentation/data-sheets/1930f.pdf}\(>\)

\bibitem{MCP1640_dtsh}
    Microchip Technology Inc.:
    \emph{MCP1640/B/C/D}\/ [online].
    2010, [cit. 30.\,12.\,2022].
    Dostupné z~URL: 
    \(<\)\url{https://ww1.microchip.com/downloads/aemDocuments/documents/APID/ProductDocuments/DataSheets/MCP1640-Family-Data-Sheet-DS20002234E.pdf}\(>\)

\bibitem{vib_motor_dtsh}
    LEADER:
    \emph{PRODUCT SPECIFICATION LCM1020A2945F}\/ [online].
    2021, poslední aktualizace 20.\,08.\,2021 [cit. 9.\,11.\,2022].
    Dostupné z~URL: 
    \(<\)\url{https://datasheet.lcsc.com/lcsc/2109230030_LEADER-LCM1020A2945F_C2891560.pdf}\(>\)

\bibitem{conv_cap_but_AT42QT1070_dtsh}
    Atmel:
    \emph{Atmel AT42QT1070}\/ [online].
    2013, poslední aktualizace 05.\,2013 [cit. 9.\,11.\,2022].
    Dostupné z~URL: 
    \(<\)\url{https://ww1.microchip.com/downloads/en/DeviceDoc/Atmel-9596-AT42-QTouch-BSW-AT42QT1070_Datasheet.pdf}\(>\)

\bibitem{WS2812C_dtsh}
    Worldsemi:
    \emph{WS2812C Intelligent control LED}\/ [online].
    2007, poslední aktualizace 2007 [cit. 10.\,11.\,2022].
    Dostupné z~URL: 
    \(<\)\url{https://datasheet.lcsc.com/lcsc/1810231210_Worldsemi-WS2812C_C114587.pdf}\(>\)

\bibitem{Bezdrat_muni}
    RNDr. Michal Černý, Ph.D.:
    \emph{Bezdrátové protokoly - základní přehled}\/ [online].
    2014, poslední aktualizace 16.\,01.\,2014 [cit. 12.\,11.\,2022].
    Dostupné z~URL: 
    \(<\)\url{https://is.muni.cz/el/1421/jaro2013/VIKMB15/um/Bezdratove_protokoly.pdf}\(>\)

\bibitem{ZigBee_smart}
    Smart-switch:
    \emph{ZIGBEE VS WIFI, CO JE LEPŠÍ?}\/ [online].
    2021, poslední aktualizace 10.\,03.\,2021 [cit. 13.\,11.\,2022].
    Dostupné z~URL: 
    \(<\)\url{https://www.smart-switch.cz/blog/zigbee-vs-wifi-co-je-lepsi/}\(>\)

\bibitem{LiFePO4_smart}
    VANDA, D.:
    \emph{LiFePO4 baterie: V čem jsou lepší než Li-Ion či Li-Pol a proč je chtít?}\/ [online].
    2022, poslední aktualizace 05.\,10.\,2022 [cit. 05.\,12.\,2022].
    Dostupné z~URL:
    \(<\)\url{https://insmart.cz/lifepo4-baterie-v-cem-jsou-lepsi-nez-li-ion-ci-li-pol-a-co-nabizi/}\(>\)

\bibitem{LiFePO4_malina}
    MALINA GROUP:
    \emph{Co jsou to baterie LiFePO4?}\/ [online].
    2021, poslední aktualizace 27.\,11.\,2021 [cit. 05.\,12.\,2022].
    Dostupné z~URL:
    \(<\)\url{https://malinagroup.cz/co-jsou-to-baterie-lifepo4/?gclid=Cj0KCQiAyracBhDoARIsACGFcS5eip8JqXIovxZ4ZCmRtD1Qhd0keRIml-H54afd2dTpAnDb95mwp1saAqaxEALw_wcB}\(>\)

\bibitem{olovene}
    ŠPINA, M.:
    \emph{Olověné baterie: Stálice na poli akumulace již více než půldruhého století}\/ [online].
    2021, poslední aktualizace 17.\,06.\,2021 [cit. 05.\,12.\,2022].
    Dostupné z~URL:
    \(<\)\url{https://oenergetice.cz/akumulace-energie/olovene-baterie-stalice-poli-akumulace-jiz-vice-nez-puldruheho-stoleti}\(>\)

\bibitem{Li-Ion}
    DUFKOVÁ, M.:
    \emph{Li-ion baterie}\/ [online].
    2015, poslední aktualizace 25.\,04.\,2015 [cit. 05.\,12.\,2022].
    Dostupné z~URL:
    \(<\)\url{https://www.3pol.cz/cz/rubriky/prakticke-informace/1677-li-ion-baterie}\(>\)

\bibitem{akumulatory}
    ASTRA:
    \emph{Přehledné informace o typech akumulátorů}\/ [online].
    2018, poslední aktualizace 28.\,11.\,2018 [cit. 05.\,12.\,2022].
    Dostupné z~URL:
    \(<\)\url{https://www.astramodel.cz/cz/blog/prehledne-informace-o-typech-akumulatoru.html}\(>\)

\bibitem{LoRa_IoT_PORT}
    IoTPORT:
    \emph{LoRaWAN - připojení do sítě IoT}\/ [online].
    2022, [cit. 28.\,12.\,2022].
    Dostupné z~URL:
    \(<\)\url{https://www.iotport.cz/lorawan-sit-pro-iot}\(>\)
    
\bibitem{LoRa_eman}
    PECH, J.:
    \emph{IOT TECHNOLOGIE: LORA A LORAWAN (3/5)}\/ [online].
    2019, poslední aktualizace 19.\,02.\,2019 [cit. 28.\,12.\,2022].
    Dostupné z~URL:
    \(<\)\url{https://www.eman.cz/blog/iot-technologie-lora-a-lorawan-3-5/}\(>\)

\bibitem{Mech_tl_princip}
    SSP Brno:
    \emph{Klávesnice}\/ [online].
    2019, poslední aktualizace 19.\,02.\,2019 [cit. 29.\,12.\,2022]. %datum poslední aktualizace - upravit
    Dostupné z~URL:
    \(<\)\url{https://moodle.sspbrno.cz/pluginfile.php/11562/mod_resource/content/2/cast2_06_klavesnice.pdf}\(>\)

\bibitem{civka_dtsh}
    Sumida:
    \emph{SMD Power Inductor CDRH3D18}\/ [online].
    2017, poslední aktualizace 09.\,01.\,2017 [cit. 01.\,01.\,2023]. 
    Dostupné z~URL:
    \(<\)\url{https://datasheet.lcsc.com/lcsc/1809140821_Sumida-CDRH3D18NP-4R7NC_C167273.pdf}\(>\)

\bibitem{piezo_dtsh}
    Bestar Acoustic:
    \emph{Magnetic transducer}\/ [online].
    2003, poslední aktualizace 28.\,05.\,2003 [cit. 01.\,01.\,2023]. 
    Dostupné z~URL:
    \(<\)\url{https://www.tme.eu/Document/1ff2ab27ffcd141d4c8d7506962ac351/bmt1205xh7_5_020200.pdf}\(>\)

\end{thebibliography}


%%%%%%%%%%%%%%%%%%%%%%%%%%%%%%%%%%%%%%%%%%%%%%%%%%%%%%%%%%%%%%%%%%%%%%%%%
%%2) Seznam citací pomocí BibTeXu
%% Při použití je nutné v TeXnicCenter ve výstupním profilu aktivovat spouštění BibTeXu po překladu.
%% Definice stylu seznamu
%\bibliographystyle{unsrturl}
%% Pro českou sazbu lze použít styl czechiso.bst ze stránek
%% http://www.fit.vutbr.cz/~martinek/latex/czechiso.tar.gz
%%\bibliographystyle{czechiso}
%% Vložení souboru se seznamem citací
%\bibliography{text/literatura}
%
%% Následující příkaz je pouze pro ukázku sazby literatury při použití BibTeXu.
%% Způsobí citaci všech zdrojů v souboru literatura.bib, i když nejsou citovány v textu.
%\nocite{*}