\chapter*{Úvod}
\phantomsection
\addcontentsline{toc}{chapter}{Úvod}
%přizpůsobit zadání 
Tato práce se zabývá návrhem Univerzálního modulu, který je určen pro naprogramování uživatelem. Univerzální modul slouží jako pomocník pro outdoorové 
týmové aktivity nebo pro edukační účely. Je kladen důraz na co 
nejširší možnosti využití. Může plnit funkci zástupu organizátora na stanovišti. Pomocí dotykových senzorů lze zadávat kódy a na základě správného 
zadání lze získávat potřebné informace pro dokončení úkolu. Slouží primárně pro komunikaci s~okolním prostředím a ovládání Univerzálního modulu během
aktivity. Informace o~stavu Univerzálního modulu, o~průběhu hry, o~správnosti kódu a podobně
jsou předávány pomocí světelných, zvukových a vibračních signalizací. 
Univerzální modul je řízen mikrokontrolérem. Část práce je proto věnována výběru všech těchto potřebných částí.

Univerzální modul je tedy navrhován pro outdoorové aplikace. Proto je kladen důraz na kompaktnost, bezpečnost, nízkou cenu, nízkou spotřebu a jednoduchost, 
aby s~ním mohl manipulovat pouhý laik, který má k~tomuto zařízení návod. Celý návrh je také koncipován tak, aby bylo možné zaručit voděodolnost 
výsledného zařízení. 

Základními požadavky na funkci Univerzálního modulu je, aby mohl svítit různými barvami. Proto je část práce věnována právě výběru součástek pro světelnou 
signalizaci. Dotykové senzory jsou vybírány dle požadavku na spolehlivost, cenu a~možnost voděodolnosti. Outdoorové aktivity bývají často časově náročné, 
a proto byl také kladen důraz na způsob napájení. 

Dalším požadavkem pro snazší manipulaci s~Univerzálním modulem je, aby jeho veškerá nastavení mohla probíhat bezdrátově. Práce se proto věnuje také možnostem 
bezdrátové konfigurace a následné realizaci. Zároveň musí být konfigurace jednoduchá a intuitivní, aby mohl kdokoli nastavit konrétní hru, její parametry a 
vše bez problémů spustit.  

Kvůli outdoorovému použití je řešen obal elektroniky Univerzálního modulu, aby byl voděodolný. 
%dopsat 





