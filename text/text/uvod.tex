\chapter*{Úvod}
\phantomsection
\addcontentsline{toc}{chapter}{Úvod}

Tato práce se zabývá návrhem zařízení Semafor. Semafor slouží pro táborové hry a edukační účely. Plní funkci například zástupu organizátora 
na stanovišti. 
Semafor je tedy navrhován pro outdoorové aplikace. Je také kladen důraz pro jednoduchost, aby s ním mohl manipulovat pouze laik, který má 
k tomuto zařízení návod. Při návrhu je tedy kladen důraz na kompaktnost, bezpečnost a nízkou cenu. 

Základními požadavky na funkci Semaforu je, aby mohl svítit několika barvami, aby měl zvukovou signalizaci a aby bylo možné jej dotykem ovládat.
Dalším požadavkem pro rozšíření možností her je, aby Semafory mohly komunikovat mezi sebou. 

Semafory musí být také jednoduše konfigurovatelné, aby mohl běžný laik nastavit konkrétní hru, její parametry a vše bez problémů spustit. 



