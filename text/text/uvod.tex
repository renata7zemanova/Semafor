\chapter*{Úvod}
\phantomsection
\addcontentsline{toc}{chapter}{Úvod}

Tato práce se zabývá návrhem zařízení Semafor. Semafor slouží jako pomocník pro táborové hry nebo pro edukační účely. Je kladen důraz na co 
nejširší možnosti využití. Může plnit funkci například zástupu organizátora na stanovišti nebo fáborku pro určení směru, kam jít, a mnoho 
dalších. Pomocí dotykových senzorů lze zadávat kódy a na základě správného zadání lze získávat potřebné informace pro dokončení úkolu. Dotykové
senzory také slouží pro konfiguraci a nastavení funkce samotného semaforu. Informace o stavu Semaforu, o průběhu hry, o správnosti kódu a podobně
jsou předávány pomocí světelných, zvukových a vibračních signalizací. 

Semafor je tedy navrhován pro outdoorové aplikace. Proto je kladen důraz na kompaktnost, bezpečnost, nízkou cenu, nízkou spotřebu a jednoduchost, 
aby s ním mohl manipulovat pouhý laik, který má k tomuto zařízení návod. Celý návrh je také koncipován tak, aby bylo možné zaručit voděodolnost 
výsledného zařízení. 

Základními požadavky na funkci Semaforu je, aby mohl svítit různými barvami, proto je část práce věnována právě výběru součástek pro světelnou 
signalizaci. Dotykové senzory jsou vybírány dle požadavku na spolehlivost, cenu a možnost voděodolnosti.

Dalším požadavkem pro rozšíření možností her je, aby Semafory mohly komunikovat mezi sebou. Mohou si tak předávat informace o aktuální svítící 
barvě nebo stisknutém tlačítku. 

Semafor je řízen mikrokontrolérem, proto je část práce věnována jeho výběru. Nemalá část je také věnována výběru typu napájení Semaforu vzhledem 
k použití na táborech a venkovních akcích.  

Semafory musí být také jednoduše konfigurovatelné, aby mohl běžný laik nastavit konkrétní hru, její parametry a vše bez problémů spustit. 



