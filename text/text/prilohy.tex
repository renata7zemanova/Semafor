\chapter{Schéma zapojení elektroniky}

\begin{figure}[!h]
	\begin{center}
	  \includegraphics[scale=1.2]{obrazky/blokove_schema_finalni_verze_priloha.png}
	\end{center}
	\caption[Blokové schéma zapojení elektroniky Univerzálního modulu]{Blokové schéma zapojení elektroniky Univerzálního modulu.}
\end{figure}

\includepdf[pages=1]{prilohy/schema_zapojeni}
\includepdf[pages=2]{prilohy/schema_zapojeni}
\includepdf[pages=1]{prilohy/KiCad_PCB}

\chapter{Výrobní podklady}

\begin{figure}[!h]
	\begin{center}
	  \includegraphics[scale=1.1]{obrazky/Vyrobni_podkady_F_Cu.jpg}
	\end{center}
	\caption[Přední strana DPS - Cu]{Přední strana DPS - Cu.}
\end{figure}

\begin{figure}[!h]
	\begin{center}
	  \includegraphics[scale=1.1]{obrazky/Vyrobni_podkady_B_Cu.jpg}
	\end{center}
	\caption[Zadní strana DPS - Cu]{Zadní strana DPS - Cu.}
\end{figure}

\begin{figure}[!h]
	\begin{center}
	  \includegraphics[scale=1.1]{obrazky/Vyrobni_podkady_F_Mask.jpg}
	\end{center}
	\caption[Přední strana DPS - Maska]{Přední strana DPS - Maska.}
\end{figure}

\begin{figure}[!h]
	\begin{center}
	  \includegraphics[scale=1.1]{obrazky/Vyrobni_podkady_F_Silkscreen.jpg}
	\end{center}
	\caption[Přední strana DPS - Popisky]{Přední strana DPS - Popisky.}
\end{figure}

\chapter{Vyrobené DPS}

\begin{figure}[!h]
	\begin{center}
	  \includegraphics[scale=0.23]{obrazky/DPS_final_vrchni.jpg}
	\end{center}
	\caption[Vrchní DPS]{Vrchní DPS.}
\end{figure}

\begin{figure}[!h]
	\begin{center}
	  \includegraphics[scale=0.22]{obrazky/DPS_final_spodni.jpg}
	\end{center}
	\caption[Spodní DPS]{Spodní DPS.}
\end{figure}

\chapter{Kompletní Univerzální modul}

%fotky - kompletni DPS v obalu

\chapter{Uživatelský manuál}
Univerzální modul je zapnut pomocí posuvného vypínače. Zapnutí je indikováno rozsvícením LED uprostřed Univerzálního modulu. Při zapínání Univerzálního modulu nesmí být stisknuto žádné tlačítko. Po zapnutí je načten 
poslední používaný mód. 

Po dobu 5 minut od zapnutí je možné přepnout Univerzální modul do konfiguračního módu. Tento mód je k~dispozici pomocí současného stisku horního a~spodního tlačítka. Konfigurační mód je signalizován blikáním všech 
LED modře. Po zapnutí tohoto módu je zapotřebí telefon 
nebo notebook, který má možnost připojit se k~WiFi síti. V~zařízení je zapotřebí najít WiFi síť s~názvem UniverzalniModul. Připojení proběhne po zadání hesla univerzalnimodul. Po připojení k~této WiFi síti přejďete 
do internetového vyhledávače. Do něj napište IP adresu Univerzálního modulu, tj.~192.168.4.1. 

Zobrazí se webová stránka s~konfiguračním menu Univerzálního modulu. Zde si můžete najít konkrétní hru i s~jejím popisem. Pokud lze u~dané hry nastavit nějaké parametry, jako je například počet hrajících týmů apod., tak 
jsou u~této hry místa, kam lze daný parametr vyplnit. Pro nastavení dané hry stiskněte tlačítko dané hry. V~tuto chvíli se spustí daný mód na všech Univerzálních modulech, 
které jsou v~danou chvíli zapnuty, v~blízkosti konfigurovaného Univerzálního modulu a neuběhlo u~nich 5 minut od jejich zapnutí. Po restartu konfigurovaného Univerzálního modulu je i~u~něj spuštěn nastavený mód. 

Tímto způsobem lze konfigurovat až 9 Univerzálních modulů současně. 

\chapter{Příklad outdoorové aktivity s~využitím \\Univerzálního modulu}
\section{Lost Heaven - Využití módu Vábnička}
Jedná se o jednoduchou noční přebíhací hru.

Hraje se v malých týmech na hřišti tvořeném kruhem, po jehož obvodu jsou v~pravidelných intervalech zřízeny stanoviště s Univerzálním modulem v módu Vábnička bez černé barvy, odhazovací miskou a losovacím pytlíkem. 

Ve středu kruhu je světelně označené stanoviště bez Univerzálního modulu (např. čelovkou) s odhazovací miskou. Mimo herní kruh je umístněna a světelně vyznačena policejní stanice pro “výslech” opatřená odhazovací miskou. 

Hra trvá cca 30 - 40 minut. Cílem hry je doručit co nejvíce zásilek. Zásilky tvoří malé důkazní lístky s označením cíle - barevným symbolem mafiánské rodiny, které mají být doručeny. Na zadní stranu lístku lepí mikrotýmy 
svoji důkazní samolepku. 

Mikrotým (3 - 5 lidí) se v herním poli pohybují v autíčkách. Jde o konstrukci z kartonů - ohrádku z cca 30 cm vysokých dílů, spojených do tvaru obdélníku cca 1,5 $\times$ 0,75 m, opatřenou barevným 
lakem, světlomety z baterek a zadními světly z~červených blikaček. Tým musí být v autíčku, které s sebou vláčí (za pomoci madel, nebo šlí). 

Autíčka přejíždí mezi Univerzálními moduly a to tak, že se snaží doručit zásilku na Univerzální modul, na kterém právě svítí barva dané zásilky (svítí barvou jejího barevného symbolu). Po doručení odhodí tým zásilku do 
připravené misky, stiskne středové tlačítko Univerzálního modulu a na něm se změní barva. Zároveň si tým vezme z neprůhledného pytlíku novou zásilku, kterou ihned označí svou samolepkou. Pokud tým ztratí zájem o převoz 
zásilky, může ji odevzdat ve středu hřiště do speciální nádoby. Za takto zahozenou zásilku je tým penalizován - ztrácí 0,2 výsledných bodů. Za každou doručenou zásilku se týmu přičítá 1 bod. Týmy nejsou kontrolovány 
v~souladu barvy zásilky a svitu Univerzálního modulu, hra předpokládá vysoký fair play. 

Autíčka gangsterů se snaží zasáhnout policejní vozy (s modrými majáčky - chemky na čepicích), a to střelbou barevnými papírovými koulemi - opatřenými chemkami. Autíčko je zasaženo, pokud je trefen kterýkoli hráč nebo 
konstrukce vozítka. Trefené vozítko předá policii zásilku a odjíždí na výslech na stanici (vzdálenou polovinu šířky herního pole, umístněnou mimo herní kruh). Z výslechu se okamžitě vrací do hry (jde jen o trestný doběh). 
Zabavené zásilky se započítávají jako -1 finální bod. 

\newpage
\section{Monopoly - Využití módu Klasický semafor}
Jde o stavění skřítčích domečků doplněné o obchody jako ve hře Monopoly a běhání.

Hraje se ve středně velkých týmech, cca 90 minut čistého herního času. Hra stimuluje týmovou práci, kooperaci, dělení rolí, mikromanagement, kreativitu, funkční komunikaci a dovednosti sebeprosazení, taktiky a strategického 
uvažování. Herní prostor je kruh, velký v závislosti na délce oběhu kružnice, určené podle možností hráčů, cca 100 - 200 metrů.

Cílem hry je vydělat co nejvíc peněz, během jasně daného časového limitu. Základním mechanismem jejich zisku je oběhnutí kola - vyznačeného v terénu (odměna je 2000, během plynutí herního času se může mírně zvyšovat). Délku 
okruhu, který trvá oběhnout např. 45 s,  je potřeba oběhnout pod minutu (pro splnění podmínky odměny). Každý tým může mít na trati vždy jen jednoho běžce, nicméně nutně, v každém obíhacím kole musí nasadit svého běžce. Jednotlivá 
kola (starty) se konají s mírným časovým odstupem (např. pokud je třeba oběhnout do minuty, konají se oběhy co 4 minuty).

Týmy nakupují parcely za 10 000. Parcely jsou rovnoměrně umístěné na trati – jsou to cedule s názvem stavby, která se zde dá postavit. U každé parcely je Univerzální modul v módu Klasického semaforu a odpočet je nastaven v intervalu 
10 až 15 s. 
Pokud je na něm červená, musí se zde všichni probíhající zastavit a čekat na zelenou. Lze také zaplatit poplatek za průjezd – (do připraveného kelímku se odhodí bankovka). Pokud hráč zaplatí, nečeká na zelenou. Hra vyžaduje vysokou 
míru fair play. 

Na sloupky zakoupených parcel se zavěsí šátek týmu. Od této chvíle poplatky za průjezd vybírá vlastník. Týmy na svých parcelách staví z přírodnin, nebo z lega, domečky – stavby, které jsou zde vyžadovány. Po postavení mohou přivolat 
kolaudační 
úředníky. Na základě kolaudace se určí, jak dobrá stavba je, a kolik se bude platit za její průjezd. Na začátku hry dostane každý tým 2 000. Stavby lze modernizovat, přivoláním kolaudačního úředníka, a zvýšit tak poplatky za průjezd. 

Na konci hry se vrátí týmům také prostředky za nákup parcel a to 10 000 za nejlepší stavbu, 8 000 za středně dobrou stavby a 5 000 nejhorší zkolaudovanou stavbu. Pokud tým neosadí parcelu stavbou, peníze se nevrací. 


\newpage
\section{Elektrický ohradník - Využití módu Odpočítávání}
Jde o hru, kde je cílem natáhnout elektrický ohradník. Hra se hraje v lese mezi stromy.

Dva členové týmu ovládají technický vozík (např. dvě páskou stažené obruče u každé je připoután jeden kbelík) Tito členové týmu jsou uvnitř tohoto stroje a~pohybují se pouze spolu s ním. Ostatní členové týmu plní několik úkolů - vytváří 
před vozíkem plot a staví cestičku z větviček, které nesmí být větší než 60 cm (nutno odměřit měrnou větví). 

Hráči ve vozíku se pohybují pouze po plotu, po této cestičce, a to tak, že zadní hráč (mechanik) háže šišky, kterými jeho kbelíky zásobují zbylí hráči v týmu, na stromy před sebou. Pokud se trefí do stromu, řidič (přední hráč) jej upozorní 
a~popojedou až na trefené místo. Takto se nahazují kabely. Jednou za 2 minuty je potřeba vyměnit posádku vozíku - bezpečnostní přestávka v práci. Jednou za 3~minuty je potřeba doplnit palivo - k vozíku musí být přinesen barel ze startu 
hry. Prázdný barel musí zpět na start. Dojde-li vozíku palivo, ztrácí tým spojení na pěti posledních stromech, tzn. vozík se vrací o 5 stromů dozadu a musí je nahodit znovu.

Tým se snaží, aby po uplynutí časového limitu měl nahozeným ohradníkem spojeno co nejvíc stromů v herním prostoru. Za spojený se považuje strom, od nějž je trať nejvýše 3 stopy, a kterého se dotkl mechanik ve vozíku. 

\iffalse
\section{Příkazy pro sazbu veličin a jednotek}

\begin{table}[!h]
  \caption[Přehled příkazů]{Přehled příkazů pro matematické prostředí }
  \begin{center}
  	\small
	  \begin{tabular}{|c|c|c|c|}
	    \hline
	    Příkaz    						& Příklad 					& Zdroj příkladu  							& Význam  \\
	    \hline\hline
	    \verb|\textind{...}|	& $\beta_\textind{max}$ 	& \verb|$\beta_\textind{max}$|	& textový index \\
	    \hline
	    \verb|\const{...}| 		& $\const{U}_\textind{in}$ 				& \verb|$\const{U}_\textind{in}$|		& konstantní veličina \\
	    \hline
	    \verb|\var{...}| 		& $\var{u}_\textind{in}$ & \verb|$\var{u}_\textind{in}$| & proměnná veličina \\
	    \hline
	    \verb|\complex{...}| 	& $\complex{u}_\textind{in}$ & \verb|$\complex{u}_\textind{in}$| & komplexní veličina \\
	    \hline
	    \verb|\vect{...}| 		& $\vect{y}$ 						& \verb|$\vect{y}$| & vektor \\
	    \hline
	    \verb|\mat{...}| 	& $\mat{Z}$ 						& \verb|$\mat{Z}$| & matice \\
	    \hline
	    \verb|\unit{...}| 		& $\unit{kV}$ 						& \verb|$\unit{kV}$|\quad či\ \, \verb|\unit{kV}| & jednotka \\
	    \hline
	  \end{tabular}
  \end{center}
\end{table}



\newpage
\section{Příkazy pro sazbu symbolů}

\begin{itemize}
  \item
    \verb|\E|, \verb|\eul| -- sazba Eulerova čísla: $\eul$,
  \item
    \verb|\J|, \verb|\jmag|, \verb|\I|, \verb|\imag| -- sazba imaginární jednotky: $\jmag$, $\imag$,
  \item
    \verb|\dif| -- sazba diferenciálu: $\dif$,
  \item
    \verb|\sinc| -- sazba funkce: $\sinc$,
  \item
    \verb|\mikro| -- sazba symbolu mikro stojatým písmem%
			\footnote{znak pochází z~balíčku \texttt{textcomp}}: $\mikro$,
	\item
		\verb|\uppi| -- sazba symbolu $\uppi$
			(stojaté řecké pí, na rozdíl od \verb|\pi|, což sází $\pi$).
\end{itemize}

Všechny symboly jsou určeny pro matematický mód, vyjma \verb|\mikro|, jenž je\\ použitelný rovněž v~textovém módu.
$\upmikro$


\chapter{Druhá příloha}

Pro sazbu vektorových obrázků přímo v~\LaTeX{}u je možné doporučit balíček \href{https://www.ctan.org/pkg/pgf}{\texttt{TikZ}}.
Příklady sazby je možné najít na \href{http://www.texample.net/tikz/examples/}{\TeX{}ample}.
Pro vyzkoušení je možné použít programy QTikz nebo TikzEdt.




%\chapter{Příklad sazby zdrojových kódů}

%\section{Balíček \texttt{listings}}

%Pro vysázení zdrojových souborů je možné použít balíček \href{https://www.ctan.org/pkg/listings}{\texttt{listings}}.
%Balíček zavádí nové prostředí \texttt{lstlisting} pro sazbu zdrojových kódů, jako například:
%
%\begin{lstlisting}[language={[LaTeX]TeX}]
%\section{Balíček lstlistings}
%Pro vysázení zdrojových souborů je možné použít
%	balíček \href{https://www.ctan.org/pkg/listings}%
%	{\texttt{listings}}.
%Balíček zavádí nové prostředí \texttt{lstlisting} pro
%	sazbu zdrojových kódů.
%\end{lstlisting}
%
%Podporuje množství programovacích jazyků.
%Kód k~vysázení může být načítán přímo ze zdrojových souborů.
%Umožňuje vkládat čísla řádků nebo vypisovat jen vybrané úseky kódu.
%Např.:

%\noindent
%Zkratky jsou sázeny v~prostředí \texttt{acronym}:
%\label{lst:zkratky}
%\lstinputlisting[language={[LaTeX]TeX},nolol,numbers=left, firstnumber=6, firstline=6,lastline=6]{text/zkratky.tex}
%
%Šířka textu volitelného parametru \verb|KolikMista| udává šířku prvního sloupce se zkratkami.
%Proto by měla být zadávána nejdelší zkratka nebo symbol.

%\shorthandoff{-}
%\lstinputlisting[language={[LaTeX]TeX},frame=single,caption={Ukázka sazby zkratek},label=lst:symfvz,numbers=left,linerange={bsymfvz-\%\%\%\ esymfvz},includerangemarker=false]{text/zkratky.tex}
%\shorthandon{-}

%\noindent
%Ukončení seznamu je provedeno ukončením prostředí:
%\lstinputlisting[language={[LaTeX]TeX},nolol,numbers=left,firstnumber=26,linerange=26]{text/zkratky.tex}

%\vspace{\fill}

%\noindent
%{\bf Poznámka k~výpisům s~použitím volby jazyka \verb|czech| nebo \verb|slovak|:}\newline
%Pokud Váš zdrojový kód obsahuje znak spojovníku \verb|-|, pak překlad může skončit chybou.
%Ta je způsobená tím, že znak \verb|-| je v~českém nebo slovenském nastavení balíčku \verb|babel| tzv.\ aktivním znakem.
%Přepněte znak \verb|-| na neaktivní příkazem \verb|\shorthandoff{-}| těsně před výpisem a hned za ním jej vraťte na aktivní příkazem \verb|\shorthandon{-}|.
%Podobně jako to je ukázáno ve zdrojovém kódu šablony.


%\clearpage

%\section{Výpis kódu prostředí Matlab}
%Na výpisu \ref{lst:priklad.vypis.kodu.Matlab} naleznete příklad kódu pro Matlab, na výpisu \ref{lst:priklad.vypis.kodu.C} zase pro jazyk~C.

%\lstnewenvironment{matlab}[1][]{%
%\iflanguage{czech}{\shorthandoff{-}}{}%
%\iflanguage{slovak}{\shorthandoff{-}}{}%
%\lstset{language=Matlab,numbers=left,#1}%
%}{%
%\iflanguage{slovak}{\shorthandon{-}}{}%
%\iflanguage{czech}{\shorthandon{-}}{}%
%}

%\begin{matlab}[frame=single,float=htbp,caption={Příklad Schur-Cohnova testu stability v~prostředí Matlab.},label=lst:priklad.vypis.kodu.Matlab,numberstyle=\scriptsize, numbersep=7pt]
%% Priklad testovani stability filtru

% koeficienty polynomu ve jmenovateli
%a = [ 5, 11.2, 5.44, -0.384, -2.3552, -1.2288];
%disp( 'Polynom:'); disp(poly2str( a, 'z'))

%disp('Kontrola pomoci korenu polynomu:');
%zx = roots( a);
%if( all( abs( zx) < 1))
%    disp('System je stabilni')
%else
%    disp('System je nestabilni nebo na mezi stability');
%end

%disp(' '); disp('Kontrola pomoci Schur-Cohn:');
%ma = zeros( length(a)-1,length(a));
%ma(1,:) = a/a(1);
%for( k = 1:length(a)-2)
%    aa = ma(k,1:end-k+1);
%    bb = fliplr( aa);
%    ma(k+1,1:end-k+1) = (aa-aa(end)*bb)/(1-aa(end)^2);
%end

%if( all( abs( diag( ma.'))))
%    disp('System je stabilni')
%else
%    disp('System je nestabilni nebo na mezi stability');
%end
%\end{matlab}

%\noindent
%\begin{minipage}{\linewidth}


%\section{Výpis kódu jazyka C}

\begin{lstlisting}[frame=single,numbers=right,caption={Příklad implementace první kanonické formy v~jazyce C.},label=lst:priklad.vypis.kodu.C,basicstyle=\ttfamily\small, keywordstyle=\color{black}\bfseries\underbar,]
// první kanonická forma
short fxdf2t( short coef[][5], short sample)
{
	static int v1[SECTIONS] = {0,0},v2[SECTIONS] = {0,0};
	int x, y, accu;
	short k;

	x = sample;
	for( k~= 0; k~< SECTIONS; k++){
		accu = v1[k] >> 1;
		y = _sadd( accu, _smpy( coef[k][0], x));
		y = _sshl(y, 1) >> 16;

		accu = v2[k] >> 1;
		accu = _sadd( accu, _smpy( coef[k][1], x));
		accu = _sadd( accu, _smpy( coef[k][2], y));
		v1[k] = _sshl( accu, 1);

		accu = _smpy( coef[k][3], x);
		accu = _sadd( accu, _smpy( coef[k][4], y));
		v2[k] = _sshl( accu, 1);

		x = y;
	}
	return( y);
}
\end{lstlisting}
\end{minipage}







%\chapter{Obsah elektronické přílohy}
%Elektronická příloha je často nedílnou součástí semestrální nebo závěrečné práce.
%Vkládá se do informačního systému VUT v~Brně ve vhodném formátu (ZIP, PDF\,\dots).

%Nezapomeňte uvést, co čtenář v~této příloze najde.
%Je vhodné okomentovat obsah každého adresáře, specifikovat, který soubor obsahuje důležitá nastavení, který soubor je určen ke spuštění, uvést nastavení kompilátoru atd.
%Také je dobře napsat, v~jaké verzi software byl kód testován (např.\ Matlab 2018b).
%Pokud bylo cílem práce vytvořit hardwarové zařízení,
%musí elektronická příloha obsahovat veškeré podklady pro výrobu (např.\ soubory s~návrhem DPS v~Eagle).

%Pokud je souborů hodně a jsou organizovány ve více složkách, je možné pro výpis adresářové struktury použít balíček \href{https://www.ctan.org/pkg/dirtree}{\texttt{dirtree}}.

%\bigskip

%{\small
%
%\dirtree{%.
%.1 /\DTcomment{kořenový adresář přiloženého archivu}.
%.2 logo\DTcomment{loga školy a fakulty}.
%.3 BUT\_abbreviation\_color\_PANTONE\_EN.pdf.
%.3 BUT\_color\_PANTONE\_EN.pdf.
%.3 FEEC\_abbreviation\_color\_PANTONE\_EN.pdf.
%.3 FEKT\_zkratka\_barevne\_PANTONE\_CZ.pdf.
%.3 UTKO\_color\_PANTONE\_CZ.pdf.
%.3 UTKO\_color\_PANTONE\_EN.pdf.
%.3 VUT\_barevne\_PANTONE\_CZ.pdf.
%.3 VUT\_symbol\_barevne\_PANTONE\_CZ.pdf.
%.3 VUT\_zkratka\_barevne\_PANTONE\_CZ.pdf.
%.2 obrazky\DTcomment{ostatní obrázky}.
%.3 soucastky.png.
%.3 spoje.png.
%.3 ZlepseneWilsonovoZrcadloNPN.png.
%.3 ZlepseneWilsonovoZrcadloPNP.png.
%.2 pdf\DTcomment{pdf stránky generované informačním systémem}.
%.3 student-desky.pdf.
%.3 student-titulka.pdf.
%.3 student-zadani.pdf.
%.2 text\DTcomment{zdrojové textové soubory}.
%.3 literatura.tex.
%.3 prilohy.tex.
%.3 reseni.tex.
%.3 uvod.tex.
%.3 vysledky.tex.
%.3 zaver.tex.
%.3 zkratky.tex.
%.2 navod-sablona\_FEKT.pdf\DTcomment{návod na používání šablony}.
%.2 sablona-obhaj.tex\DTcomment{hlavní soubor pro sazbu prezentace k~obhajobě}.
%.2 readme.txt\DTcomment{soubor s~popisem obsahu CD}.
%.2 sablona-prace.tex\DTcomment{hlavní soubor pro sazbu kvalifikační práce}.
%.2 thesis.sty\DTcomment{balíček pro sazbu kvalifikačních prací}.
%}
%}

\fi