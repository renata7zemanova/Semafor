\chapter{Požadavky}
voděodolnost


\chapter{Základní návrh}
Základní návrh se skládá především z výběru bezdrátové komunikace, která je klíčová. Díky ní budou moci Semafory komunikovat mezi 
sebou, takže si například budou moci předávat informace o barvě, kterou svítí nebo stisku tlačítek apod. 

Další nedílnou součástí je mikrokonrolér, který řídí veškerou činnost každého Semaforu. U Semaforu je také nutné řešit způsob napájení,
takže i to je součástí návrhu.

Celkový návrh obsahuje také výběr tlačítek, zvukových a vizuálních signalizací a potřebných převodníků. 

%bude tady také blokové schéma

\section{Bezdrátová komunikace}
Použití na táborech a outdoorových akcích vyřadilo z výběru drátovou komunikaci. K DPS by muselo být ještě velké množství kabelů 
o délce několika stovek metrů minimálně. Bezdrátová komunikace je z tohoto hlediska velmi praktická. Je to také moderní řešení 
náležící dnešní době. 

Jedním ze základních požadavků bylo, že jednotlivé DPS mezi sebou musí být schopny komunikovat. Vzhledem k použití na táborech 
musel být vybrán komunikační protokol a následně k němu přizpůsoben hardware. 

Práce tedy započala tím, že byla udělána rešerše existujících bezdrátových komunikačních protokolů a následně byly tyto protokoly 
mezi sebou porovnány. Vzhledem k použití na outdoorových akcí byl kladem důraz na komunikační vzdálenost a náročnost na výkon, 
jelikož zařízení je napájeno z baterií. 

Dalším požadavkem bylo bezdrátové nastavování, tedy připojení k Semaforu např. přes telefon a odeslání konfigurace. Nastavení hry,
která se hraje a např. čas, jak dlouho se bude hrát, nebo v kolika týmech, je tedy zapotřebí také dělat bezdrátově. 

\subsection{WiFi}
Komunikace pomocí WiFi sítě je jednou z nejznámějších a nejpoužívanějších bezdrátových komunikací užívaných širokou veřejností. 
WiFi je dnes na každém pracovišti, na veřejných místech i v každé domácnosti. Využívána je především k připojení k internetu. 
Přes WiFi lze přenášet velké objemy dat vysokou rychlostí. Pracuje v pásmech v okolí frekvencí 2,4 GHz a 5,0 GHz s dosahem 
desítek až nižších stovek metrů \cite{Bezdrat_muni}.
%doplnit

Výhody bezdrátové technologie WiFi jsou \cite{Bezdrat_muni}:
\begin{itemize}
  \item pracuje v bezlicenčním pásmu, %doplnit do textu
  \item levná, %doplnit do textu
  \item velmi rozšířená.
\end{itemize}

Nevýhody jsou \cite{Bezdrat_muni}:
\begin{itemize}
  \item omezený výkon (není možné pokrýt rozáhlejší oblasti), %doplnit do textu
  \item vyšší spotřeba energie.
\end{itemize}
%doplnit

\subsection{Bluetooth}
Bluetooth je také velmi rozšířenou technologií bezdrátové komunikace. Používá se na přenos dat na krátké vzdálenosti. V dnešní době 
rozšířené WiFi komunikace je její použití omezené. Běžně se využívala pro přenos fotografií z jednoho zařízení do druhého apod. V dnešní 
době se spíše využívá pro připojení bezdrátových periferií jako jsou bezdrátová sluchátka, myši a klávesnice. Tato technologie je zaměřena 
především na nízkou spotřebu, i proto je komunikační vzdálenost maximálně 100 metrů \cite{Bezdrat_muni}. V praxi jde ale o nižší desítky
metrů. Bluetooth je také technologií pro propojení pouze 2 zařízení, kde jedno je tzv. master a druhý tzv. slave \cite{Bezdrat_muni}. 
Jedno zařízení je tedy nadřazeno druhému. V případě telefonu a sluchátek je telefon nadřazený sluchátkům. 
%doplnit

Výhody bezdrátové technologie Bluetooth jsou \cite{Bezdrat_muni}:
\begin{itemize}
  \item nízká spotřeba.
\end{itemize}

Nevýhody jsou \cite{Bezdrat_muni}:
\begin{itemize}
  \item krátký dosah,
  \item možnost propojení pouze 2 zařízení.
\end{itemize}
%doplnit

\subsection{NFC}
NFC je jedna z novějších technologií, která je známá především při použití platby kartou. Jde tedy o přenos malých objemů dat na velmi krátkou 
vzdálenost, tj. do desítek centimetrů \cite{Bezdrat_muni}. NFC je technologií, kde stačí, aby pouze jedno zařízení mělo zdroj elektrické 
energie \cite{Bezdrat_muni}. Druhé zařízení se chová jako anténa, ze které je možné vyčíst informace \cite{Bezdrat_muni}. Například při 
platbě kartou v sobě karta nemá žádný zdroj energie, ale při přiložení k terminálu je pomocí elektromagnetické indukce vyčteno identifikační
číslo karty. Díky tomu je možné zaplatit. 
%doplnit

Výhody bezdrátové technologie NFC jsou \cite{Bezdrat_muni}:
\begin{itemize}
  \item rychlost, %doplnit do textu
  \item možnost interakce se zařízeními bez vlastního zdroje elektrické energie.
\end{itemize}

Nevýhody jsou \cite{Bezdrat_muni}:
\begin{itemize}
  \item velmi krátká komunikační vzdálenost,
  \item možnost komunikace pouze mezi dvěma zařízeními, 
  \item nízká rychlost přenosu,
  \item malý objem přenášených dat.
\end{itemize}
%doplnit
%doplnit do zkratek NFC (Near Field Communication), WiFi (Wireless Fidelity)

\subsection{ZigBee} %mozna cele smazat
ZigBee technologie je používána pro vytvoření malých sítí, kde může signál snadno přeskakovat z jednoho zařízení na druhé \cite{ZigBee_smart}.
Není přitom zapotřebí, aby bylo každé zařízení připojeno k internetu pomocí WiFi \cite{ZigBee_smart}. Pro komunikaci je ale zapotřebí centrální 
rozbočovač, kterž zajišťuje komunikaci mezi zařízeními \cite{ZigBee_smart}. Tato technologie je určena pro tvorbu rozsáhlejších bezdrátových sítí
s přenosem menšího objemu dat \cite{ZigBee_smart}. Jedná se o spolehlivou technologii s nenáročnou implementací a nízkou spotřebou elektrické energie 
\cite{ZigBee_smart}. Díky ZigBee může mít uživatel v jedné aplikaci zařízení 
od různých značek a výrobců, protože právě ZigBee zajišťuje jejich vzájemnou komunikaci \cite{ZigBee_smart}.

Technologie ZigBee je určeno primárně pro senzorové sítě v průmyslových aplikacích \cite{Bezdrat_muni}. Není vhodný pro práce s velkými objemy dat \cite{Bezdrat_muni}.
Pracuje v bezlicenčním frekvenčním pásmu \cite{Bezdrat_muni}.
%doplnit

%Všechna zařízení jsou schopna komunikovat a být ovládána jedinou jednotkou.

Výhody bezdrátové technologie ZigBee jsou \cite{ZigBee_smart}:
\begin{itemize}
  \item nízká spotřeba elektrické energie,
  \item spolehlivost, 
  \item nenáročná implementace,
  \item pracuje v bezlicenčním frekvenčním pásmu. 
\end{itemize}

Nevýhody jsou \cite{ZigBee_smart}:
\begin{itemize}
  \item nutnost centrálního rozbočovače.
\end{itemize}
%doplnit

\subsection{LoRa}
%co to je, 

%výhody

%Velká přehledová tabulka s důležitými údaji od každého protokolu, aby bylo jasné, proč byla vybrána LoRa.

\subsection{Výběr bezdrátové technologie}
Ke komunikaci Semaforů mezi sebou byla zvolena technologie LoRa. Tato technologie byla zvolena především kvůli komunikačnímu dosahu. Jedná se sice o dražší technologii, 
ale na tolik, aby ji nebylo možné v tomto zařízení použít. Táborové hry se většinou hrají na loukách, které mají rozlohu několik stovek metrů čtverečných. LoRa je jedinou
dostupnou technologií, která na tyto vzdálenosti spolehlivě komunikuje. Bezdrátové propojení Semaforů bude použito pro posílání informací o aktuálně svítící barvě, či 
stisku tlačítka v závislosti na hře, která se aktuálně hraje. U některých her například může být žádoucí, aby po přepínání nesvítily všechny Semafory stejnou barvou,
díky této komunikaci se bude moci být takovým stavům zabráněno. 

K propojení Semaforu s telefonem a dalšími zařízeními byla vybrána technologie WiFi. Jedná se o rozšířenou technologii, která je v telefonech a noteboocích zabudovaná. 
Propojení bude tedy jednoduché a nastavovat hry se mohou na webové stránce. Mikrokontrolér vytvoří WiFi, ke které se pomocí telefonu připojí. Po připojení bude zobrazena
webová stránka, kde bude seznam her, které Semafor umí. U jednotlivých her se poté budou moci nastavovat další parametry. Po nasatvení se komfigurace pošle do Semaforu. 

\section{Mikrokontrolér}
Faktory ovlivňující výběr řídicího mikrokonroléru:
\begin{itemize}
  \item dostatečný počet GPIO pinů,
  \item dostatek paměti,
  \item WiFi,
  \item ADC,
  \item cena,
  \item periferie pro připojení Lora modulu.
\end{itemize}

%do zkratek ZigBee, GPIO, ADC
Dostatečný počet GPIO %specifikovat minimum
Dostatek paměti %lépe specifikovat
WiFi
ADC %pro senzor pro osvětlení
Cena %rozumné hranice
Periferie pro připojení pro LoRa modul

Nice to have:
USB rozhraní (nemusel by být použit převodník)
Možnost režimu spánku %možná nepsat


Jedním z požadavků na mikrokonrolér byla zabudování WiFi. Asi nejznámnějšími mikrokonroléry s WiFi jsou ESP32 od firmy 
Espressif. Mikrokontroléry ESP32 jsou dostupné v různých variantách, ale všechny mají WiFi a dělají se i ve variantách 
se zabudovanou anténou. 
%proč jsem vybrala ESP
%proč právě tento typ 
%napsat přesné označení čipu - dále ž jen ESP32-C3

Rozsah napájecího napětí je 3 až 3,6 V \cite{ESP_C3_dtsh}.
%jaké má periferie

\section{LED}
Jendím z nejdůležitějších požadavků na Semafor bylo, aby mohl svítit. Čím více možností, jak svítit, tím bude využití 
při hrách a táborových programech různorodější. K tomuto účelu jsou použity LED. Obyčejné LED mají pouze jednu barvu, kterou 
mohou svítit. Existují také RBG LED, ale ty mají 4 vývody a každá LED tak zabírá 3 GPIO piny mikrokontroléru - jeden pin
pro jednu barvu. K tomu by byl zapotřebí mikrokonrolér s velkým množství GPIO pinů. Takových mikrokonrolérů není mnoho a zároveň 
by se to odrazilo na ceně. Počet GPIO pinů je jedním z hlavních limitujících faktorů při výběru MCU. Proto byly použity programovatelné 
LED typu WS2812C.

Tyto programovatelné LED WS2812C lze spojovat za sebe, takže datový výstup jedné LED je připojen k datovému vstupu další 
LED \cite{WS2812C_dtsh}. Takto lze spojit nekonečné množství těchto programovatelných LED a připojit je na jeden GPIO pin MCU. 
Každá LED má pin pro vstupní napětí, GND, vstupní datový pin a výstupní datový pin. Typ inteligentních LED WS2812C je vhodný pro 
bateriová zařízení. Oproti častěji používanému typu WS2812B mají 3~$\times$~ menší spotřebu elektrické energie. 

Napájecí napětí těchto LED by se mělo pohybovat v rozmezí 4,5 až 5,5 V \cite{WS2812C_dtsh}. Kondenzátor u každé LED slouží pro 
filtraci napájecího napětí. 

%způsob programování (jak funguje použitá knihovna?)

%obrázek spojení LED za sebe

Komunikační napěťová úroveň logické jedničky těchto LED by měla být alespoň na úrovni 70 \% napájecího napětí \cite{WS2812C_dtsh}. 
Protože použitý mikrokonrolér ESP32-C3 má komunikační napěťovou úroveň logické jedničky jeho napájecí napětí, což je 3 až 3,6 V, 
tak je zapotřebí využít převodník napěťové úrovně \cite{ESP_C3_dtsh}. Komunikace je v tomto případě pouze jednosměrná, 
to znamená, že MCU posílá data do LED, ale LED neposílají žádná data do MCU. Převodník je realizován unipolárním tranzistorem 
a jedním pullup rezistorem. Rezistor je připojen pro k napájecímu napětí inteligentních LED WS2812C. 
Tranzistor Q1 má gate připojený k napájecímu napětí MCU. Pokud bude mikrokonrolér do LED posílat logickou jedničku, tak bude rozdíl
mezi gate a source 0 V. Tím pádem bude tranzistor uzavřený a tím se přes rezistor R4 připojí k LED jejich napájecí napětí. Toto napětí 
je pro inteligentní LED logickou jedničkou. Pokud bude MCU posílat logickou nulu, tedy 0 V, tak je rozdíl napětí mezi gate a source 
napájecí napětí mikrokontroléru. Tranzistor je tedy otevřený a tím se napětí 0 V dostane k inteligentním LED a na rezistoru se objeví
úbytek napětí o velikosti napájecího napětí inteligentních LED. Napětí 0 V je logickou nulou i pro inteligentní LED. Tento převodník
je určen pouze pro komunikaci jedním směrem. 

%schéma převodníku 

Tyto programovatelné LED mají maximální spotřebu 5 mA na jeden kanál. Při zapnutí všech kanálů (svícení bílou) je maximální
spotřeba jedné LED 15 mA \cite{WS2812C_dtsh}. Pokud LED nesvítí, tak je její maximální klidový proud 0,3 mA \cite{WS2812C_dtsh}.
Při použití 12 LED je tedy maximální odběr všech LED 180 mA.

%do sekce o DPS se zmínit o jejich umístění - kruh, hodiny, proto 12 ks. (lze také rozdělit na 3 segmenty)



\section{Tlačítka}
Tlačítka jsou nezbytnými prvky pro ovládání Semaforu. Mohou sloužit pro přepínání módů, ovládání Semaforu jako takového nebo jako herní 
součást. Ve hře mohou plnit úlohu přepínače režimů hry, zadávání kódů, určování směru apod. 

Tlačítka mohou být realizována dvěma základními způsoby, mohou být elektromechanická, nebo dotyková kapacitní. 

Stisková plocha mechanického tlačítka je nevodivá, často plastová. Mechanické prvky jsou častým zdrojem problémů. Je tím často omezena i 
životnost celého výrobku. Mechanická konstrukce tlačítek je složitá a finančně nákladná. Mechanická tlačítka zároveň generují zákmity, které 
je nutno filtrovat nebo tvarovat do použitelné podoby. Nejjednodušším řešením je přidání kondenzátoru. Mechanická tlačítka existují typu NO 
a NC. 

Po zmáčknutí mechanického tlačítka typu NO jsou 2 kovové části tlačítka spojeny, tím dochází ke spojení elektrického obvodu 
a odpor smyčky je v ideálním případě nulový. Obvod je tedy sepnut. Když je tlačítko rozpojeno, tak je 
elektrický obvod přerušen a odpor smyčky je v ideálním případě nekonečný. Obvod je tedy rozpojen. U tlačítka typu NC je to naopak. Při stisku 
tlačítka je obvod rozepnut a při uvolnění stisku je obvod sepnut. 

%sem přidat fotku vnitřku mechanického tlačítka

Výhody mechanických tlačítek jsou:
\begin{itemize}
  \item jednoduché připojení ke každému GPIO mikrokonroléru,
  \item odezva je samotný stisk tlačítka,
  \item fyzické rozpojení obvodu.
\end{itemize}

Kapacitní tlačítka jsou bez veškerých mechanických prvků, zároveň jsou jednoduchá a mají téměř neomezenou 
životnost. Jejich výstupní signál je bez jakýchkoli zákmitů nebo rušení. Kapacitní tlačítka lze snadno použít v mnoha aplikacích. 

Kapacitní tlačítka jsou tvořena měděnou vrstvou a nejsou nijak mechanicky namáhána. Tlačítko může být zmáčknuto i přes 
obal krabičky, a proto může být celé zařízení mechanicky odolné i voděodolné. 

Nevýhodou kapacitních tlačítek je, že nemají žádnou odezvu na dotyk. U mechanických tlačítek je odezvou samotný fyzický 
stisk tlačítka. U kapacitních tlačítek lze tento fakt vyřešit například rozsvícením LED nebo vibrační odezvou. Vibrační 
odezva může být realizována pomocí vibračního motoru. 

Některé MCU včetně vybraného mikrokontroléru ESP32-C3 nemají kapacitní
vstupy, to znamená, že tlačítko nelze připojit přímo k pinu MCU \cite{ESP_C3_dtsh}. Buď musí být vybrán mikrokonrolér, který kapacitní 
vstupy má, nebo může být použit převodník, který má kapacitní vstupy a jeho výstupy poté mohou být připojeny k MCU. 

Výhody kapacitních tlačítek jsou:
\begin{itemize}
  \item kompaktnost,
  \item variabilita,
  \item vysoká spolehlivost,
  \item odolnost vůči šumu,
  \item možnost kompenzace rušivých elementů,
  \item cena. 
\end{itemize}

V návrhu Semaforu byla zvolena kapacitní dotyková tlačítka. Pro možnost použití uvnitř i venku jsou díky možnosti voděodolnosti 
vhodnějším řešením. Také velikost a označení tlačítka může být variabilní. Velikost může být na DPS navržena dle potřeby a potisk
v místě tlačítka vyznačen barevně, nebo např. samolepkou. Odezva na dotyk bude realizována pomocí vibračního motoru.

\subsection{Princip kapacitních dotykových tlačítek}
Základní princip je založen na měření změny kapacity. Měď, ze které je tlačítko vytvořeno má
nějakou vlastní kapacitu (kapacita samotné nosné desky) a po přiložení prstu je kapacita zvýšena o paralelně 
připojenou kapacitu přechodu tlačítka a prstu díky obsahu železa v krvi a vodivosti kůže \cite{PrincipKapTl}. 
Prst se tedy chová jako druhá uzemněná elektroda \cite{PrincipKapTl}. 

Kapacita snímače se tedy volí co nejmenší, aby přiložený prst vyvolal co nejvetší změnu kapacity. Ve snímači se vyskytuje
RC článek, kterého se mění doba nabíjení kondenzátoru a tím je možné detekovat stisk tlačítka \cite{PrincipKapTl}. 

\subsection{Návrh kapacitního dotykového tlačítka}
Tvar tlačítka nemá vliv na schopnost detekce dotyku \cite{PrincipKapTl}. Naopak velký vliv má plocha tlačítka, tloušťka
izolační vrstvy, a také vzdálenost jednotlivých tlačítek od sebe \cite{PrincipKapTl}. 

Čím větší je plocha tlačítka, tím je větší změna kapacity při dotyku a díky tomu je vytvořena lepší schopnost detekce 
dotyku \cite{PrincipKapTl}. S rostoucí tloušťkou izolační vrstvy se naopak schopnost detekce dotyku snižuje \cite{PrincipKapTl}.

Pokud jsou tlačítka příliš blízko u sebe, tak může docházet k jejich vzájemnému ovlivňování. Kvůli tomu pak může docházet k
detekci dotyku špatného tlačítka, nebo k falešné detekci dotyku. Z doporučení plyne, že pro dotyk prstu je vhodná velikost snímací 
plochu pro prst 13~$\times$~13 mm a jejich vzdálenost alespoň 5 mm od sebe \cite{PrincipKapTl}. Proti vzájemnému ovlivňování tlačítek
se používají uzemňovací meziplošky \cite{PrincipKapTl}. 

%sem dát obrázek, kde je ukázán návrh GND meziplošek

U kapacitních dotykových tlačítek je zapotřebí dbát na správné připojení k MCU. U vícevrstvých DPS nesmí pod tlačítky, ani pod přívody
k MCU, vést jiné dráhy, ani se zde nesmí vyskytovat jiné součástky \cite{PrincipKapTl}. Součástky nesmí být ani z vrchní, ani ze spodní 
strany DPS \cite{PrincipKapTl}. Přívody kapacitních tlačítek k MCU by měly být odstíněny pomocí GND signálu.

Voda a další nečistoty mění vlastní kapacitu tlačítka a může tak docházet k falešným stiskům tlačítka. Tento problém lze řešit softwarově. 
Lze využít faktu, že nečistoty působí dlouhodobě, ale stisk je krátkodobý \cite{PrincipKapTl}. Hodnotu vlastní kapacity tlačítka je tedy
možné softwarově upravovat v závislosti na aktuálních dlouhodobějších stavech a detekovat tak přesněji krátkodobý stisk tlačítka.

%dát do zkratek MCU, DPS, GPIO, LED, GND

%\begin{figure}[!h]
%    \begin{center}
%      \includegraphics[scale=0.5]{obrazky/ZlepseneWilsonovoZrcadloNPN}
%    \end{center}
%    \caption[Alenčino zrcadlo]{Zlepšené Wilsonovo proudové zrcadlo.}
%  \end{figure}

Pro odlišení tlačítek je místo označeno barevným potiskem. 

\section{Vibrační motor}
Vibrační motory jsou založeny na principu kmitání. Motor je připevněn k zařízení, které je kmitáním rozvibrováno. Vibrační motory jsou dnes 
nedílnou součástí mnoha elektronických zařízení včetně mobilního telefonu nebo dětských hraček. 

Dioda slouží jako ochrana proti přepětí, protože motor je indukční zátěž, takže vytváří napěťové špičky. Díky diodě je mikrokonrolér chráněn 
proti špičkovému napětí, které by se na něj mohlo dostat. Kondenzátor slouží k tomu, aby napěťové špičky eliminovat, nebo alespoň zmenšoval. 

Vibrační motor je připojen k mikrokontroléru přes tranzistor, protože maximální výstupní proud z pinu MCU není dostatečně velký na to, aby 
motor roztočil. Tranzistor je tedy připojen na gate tranzistoru, který se při logické jedničce na pinu sepne a motorem protéká proud, který 
nedodává MCU, ale zdroj 3.3 V (v tomto případě baterie LiFePO4). Baterie tak dokáže dodat dostatek proudu, aby se motor roztočil. 

Pro Semafor byl vybrán vibrační motor LCM1020A2945F. Tento motor má maximální požadovaný proud 120 mA \cite{vib_motor_dtsh}. Maximální proud, 
který lze odebírat z pinu mikrokontroléru ESP32-C3, je 40 mA \cite{ESP_C3_dtsh}. Vibrační motor lze pouze spínat, nebo je možné jej připojit 
k pinu, který dokáže generovat PWM a lze tím regulovat jeho otáčky. 

Vibrační motor slouží jako odezva na dotyk kapacitního tlačítka. 

%obrázek schéma zapojení + asi fotka vybraného motoru

\section{Převodník pro kapacitní tlačítka}
Vybraný mikrokonrolér ESP32-C3 nemá kapacitní vstupy, proto je zapotřebí kapacitní dotyková tlačítka připojit přes převodník. Je zapotřebí připojit 
5 tlačítek. 
%jaké byly možnosti 
%zmínit se i o TTP224?
%má 4 vstupy, ale já potřebuji 5 tlačítek

Použitý převodník AT42QT1070 dokáže pracovat ve 2 režimech. V prvním režimu může být zapojeno maximálně 5 kapacitních tlačítek, která jsou připojena
k pinům KEY0 až KEY4. Jako výstup se používají piny OUT0 až OUT4. Každé tlačítko má tedy svůj výstup, který může být připojen k GPIO pinům MCU 
nebo k nim mohou být připojeny např. LED \cite{conv_cap_but_AT42QT1070_dtsh}. 

Druhý režim je využitelný pouze v případě, je-li převodník připojen k MCU. Vtomto případě může být k převodníku připojeno až 7 kapacitních tlačítek, 
která jsou připojena na pinech KEY0 až KEY6. Převodník poté komunikuje s MCU pomocí komunikační sběrnice I2C \cite{conv_cap_but_AT42QT1070_dtsh}. 
Z registru převodníku lze poté vyčíst stavy daných kapacitních dotykových tlačítek. 

Jelikož je v tomto návrhu Semaforu využit mikrokontrolér, který podporuje komunikaci po sběrnici I2C, tak bylo využito právě zapojení s komunikací 
přes I2C. Díky tomu budou využity pouze 2 GPIO piny mikrokonroléru ESP32-C3 a ne 5 GPIO pinů, které by byly zapotřebí při zapojení bez komunikace pro
sběrnici I2C.

Převodník má kondenzátory C3 a C4 připojeny na napájecím pinu vůči zemi, aby nebyly případné proudové špičky přivedeny na napájení převodníku. Rezistory
R17 a R18 slouží jako pullup rezistory při komunikaci pomocí sběrnice I2C s mikrokonrolérem EP32-C3. Na piny KEY0 až KEY4 jsou připojena kapacitní 
dotyková tlačítka.  
%proč je MODE na zemi
%proč nepoužívám pin /CHANGE?
%obrázek - schéma

\section{Napájení}
Vzhledem k použití Semaforů při hrách na táborech byly možné pouze 2 způsoby napájení, pomocí powerbanky nebo baterií. Byla zvolena kombinace obou druhů.

Zabudování baterie přináší kompaktnost řešení a pro použití není třeba dalších komponent. Pokud je ale na táboře větší využití, tak se baterie vybije.
Na táborech většinou nebývá připojení k elektrické síti a proto je řešením powerbanka. Na Semaforu tedy bude napájecí vstup USB A pro nabíjení baterií
přímo z powerbanky. Semafor musí být koncipován tak, aby se mohla baterie nabíjet a zároveň, aby při tom byly Semafory funkční.

\subsection{Baterie}
Ve výběru baterií hraje velkou roli kapacita, napětí, velikost a cena. Požadavkem je také možnost nabíjení. Při použití na táboře by jinak musely být stále 
nové baterie v balení a musely by se neustále doplňovat a udržovat.

Moderní baterie jsou náchylné na přepólování, a proto není bezpečné, aby uživatel měnil baterie sám. Baterie budou tedy zabudované v zařízení bez možnosti 
výměny uživatelem. 

%kapacita baterie - jak dlouho by měla vydržet

Z nabíjecích baterií je možno vybírat z nabíjecích tužkových baterií (Ni-MH), olověné baterie, Li-Ion, Li-Pol a LiFePO4 baterií.
%Ni-MH
Baterie Ni-MH mají jmenovité napětí 1,25 V. %citace
Proto by bylo zapotřebí alespoň 3 článků spojených sériově, u kterých by navíc musel být stabilizátor
na 3,3 V pro napájení mikrokontroléru. % a co inteligentní LED?
%Olověné baterie

\subsubsection{Li-Pol}


\subsubsection{Li-Ion}


\subsubsection{LiFePO4}
LiFePO4 baterie mají jmenovité napětí v rozsahu 3 až 3,3 V \cite{LiFePO4_malina}. Její minimální provozní napětí je pak 2,5 V a maximální je 3,65 V \cite{LiFePO4_malina}.
LiFePO4 je dnes baterie známá jako nejbezpečnější, nejspolehlivější a nejstabilnější baterie obsahující lithium \cite{LiFePO4_malina}. Je to nejvhodnější baterie pro 
přenosná zařízení díky velmi dobrému poměru valikosti (hmotnosti) a kapacity. Její životnost je až 4~$\times$ vetší než u baterie Li-Ion \cite{LiFePO4_malina}. Je také hodnocena
jako nejbezpečnější z dosud dostupných baterií \cite{LiFePO4_malina}. Baterie LiFePO4 mohou standardně dosáhnout životnosti až 7000 cyklů, to odpovídá cca 15 letům při bežném
používání \cite{LiFePO4_malina}. Článek LiFePO4 je teplotně stabilní, nehořlavý (ani při zkratu), netrpí samovybíjením, není toxický a nevytéká \cite{LiFePO4_malina}.

LiFePO4 baterie v dnešní době nahrazují baterie typu Li-Ion nebo Li-Pol.
Tyto bateriové články jsou vhodné především pro použití v elektromobilech, solárních a větrných elektrárnách, elektrokoloběžkách atd. 

Výhody LiFePO4 baterií jsou \cite{LiFePO4_smart}:
\begin{itemize}
  \item vysoký jmenovitý proud,
  \item minimální ztráty,
  \item krátká doba dobíjení,
  \item chemická odolnost,
  \item vyníkající poměr výkonu ku hmotnosti,
  \item životnost tisíce cyklů,
  \item snadnější recyklace,
  \item bez využití toxických prvků.
\end{itemize}

Při realizaci Semaforu byly vybrány baterie LiFePO4 právě kvůli již zmíněným vynikajícím vlastnostem. Vybraný mikrokonrolér má napájecí napětí v rozsahu 3 až 3,6 V \cite{ESP_C3_dtsh}. 
Pro funkci mikrokonroléru tedy nebude muset být použit ani převodník napětí.  



\subsection{Nabíjecí obvod}
%nastavit nabíjecí proud na 1C 
Nabíjecí obvody jsou závislé na konkrétním typu baterií, které budou nabíjeny. Vzhledem k vybranému typu baterií LiFePO4 byly uvažovány pouze komerčně
dostupné integrované obvody, které jsou určeny pro nabíjení tohoto typu baterií. 
%ještě TP5000 zmínit? - zkontrolovat jestli je opravdu na pro LiFePO4

Vybraný typ baterií LiFePO4 lze nabíjet pomocí obvodu CN3058E \cite{charger_dtsh}. 
%možná vyjmenovat další obvody a napsat proč tento

%napětí nastaveno na 3,6 V interně, ale lze jej nastavit pomocí externího odporu
%nabíjecí proud se nastavuje externím rezistorem - není využito
%při odpojení napájecího napětí přechází tento obvod do režimu spánku, takže je baterie vybíjena proudem menším než 3 uA
%lze snímat teplotu na baterii

Nabíjecí obvod CN3058E je určen pro nabíjení pouze LiFePO4 baterií a lze jím napájet právě 1 článek těchto baterií \cite{charger_dtsh}. Napájecí napětí tohoto 
nabíjecího čipu se pohybuje mezi 3,8 až 6 V \cite{charger_dtsh}. Díky tomu lze přímo použít napětí z USB konektoru. 

%popsat další vlastnosti 

%doporučené schéma zapojení čipu - z datasheetu

Tento nabíjecí obvod se vyrábí ve standardizovaném pouzdře SOP8 \cite{charger_dtsh}.

\subsection{Zapojení nabíjecího obvodu}
Rezistor připojený k pinu ISET slouží pro nastavení hodnoty nabíjecího proudu \cite{charger_dtsh}. V tomto zapojení byl počítán pro nabíjecí proud 1 A dle rovnice: 
\begin{equation} 
  I_{CH}~=~\frac{U_{ISET}}{R_{8}}~\cdot~1011. 
  \quad \quad \quad \quad \quad \quad \cite{charger_dtsh}
\label{eq:I_CH}
\end{equation}

%1218/1 = 1218 Ohm
%rovnici + citace


%jak zjistim napeti na tom pinu?
Velikost rezistoru R8 byla počítána na velikost nabíjecího proudu 1 A dle následující rovnice:
\begin{equation} 
  R_{8}~=~\frac{U_{ISET}}{I_{CH}}~\cdot~1011~=~\frac{}{1}~\cdot~1011~=~XXX~\:k\Omega. 
  \quad \quad \quad \quad \quad \quad \cite{charger_dtsh}
\label{eq:I_CH}
\end{equation}


%tuto větu opravit
Z výpočtu vyplývá, že rezistor by měl mít hodnotu 1218 $\Omega$. Nejbližší hodnota z rezistorové řady E12 je hodnota 1,2 k$\Omega$, proto byl také zvolen rezistor 
o této hodnotě \cite{rezistorova_rada}. Odpovídá tomu nabíjecí proud 1015 mA, který nebude mít vliv na životnost baterií. 

%rovnice 1218/1200 = 1.015 A = 1015 mA

Vstupní a výstupní kondenzátory slouží pro filtaci zákmitů napájecího napětí a také napětí, kterým je nabíjena baterie. Hodnoty kondenzátorů byly převzaty
z doporučení z datasheetu.

Kladný pól nabíjené baterie je připojen na pinu BAT, záporný pól je připojen ke GND. Pin BAT poskytuje nabíjecí proud do baterie a zároveň poskytuje konstantní 
nabíjecí napětí. V režimu spánku je svodový proud tohoto pinu 3 $\mu$A \cite{charger_dtsh}. 
%do zkratek přidat GND

Pin VIN slouží pro napájení vnitřního obvodu CN3058E. Je na něj přikládáno napájecí napětí z USB, tedy 5 V. Pokud napájecí napětí klesne na napětí o 10 mV nižší, 
než je napětí na pinu BAT, tak vnitřní obvod přechází do režimu spánku \cite{charger_dtsh}. V tomto režimu klesá proud pinu BAT na méně než 3 $\mu$A \cite{charger_dtsh}.

Tento nabíjecí obvod má možnost indikace nabíjení baterií a dokončení nabíjení. Tato indikace je realizována pomocí 2 LED připojených přes pullup rezistor. Hodnota
pullup rezistoru byla převzata z doporučení z datasheetu. Červená LED indikuje nabíjení baterií a je připojena na pin /CHRG a zelená LED indikuje dokončené nabíjení 
a je připojena na pin /DONE. Obě LED jsou k pinům nabíjecího čipu připojeny katodou. 

Obvod CN3058E může také měřit teplotu na nabíjené baterii. Slouží k tomu vstupní pin TEMP. Měření probíhá pomocí odporového děliče, jehož střed je připojen na snímač 
teploty. Tento snímač je připojen na baterii. Pokud je napětí na pinu TEMP nižší než 45 \% nebo vyšší než 80 \% úrovně napájecího napětí, tak je indikována moc nízká
nebo moc vysoká teplota baterie a nabíjení je zastaveno \cite{charger_dtsh}. Jinak nabíjení pokračuje. Uzemněním pinu TEMP je funkce měření teploty deaktivována \cite{charger_dtsh}. 
V této práci není měření teploty baterií využíváno, a proto je pin TEMP připojen ke GND. 

%obrázek z datasheetu, kde je připojeno i meření teploty

Pokud není baterie nabíjena, tak by svodový proud pinu BAT nabíjecího obvodu CN3058E vybíjel baterii. Svodový proud tohoto pinu je 3 $\mu$A \cite{charger_dtsh}. 
Aby se baterie zbytečné navybíjela, tak je do obvodu připojen tranzistor Q2, který detekuje připojené napětí k nabíjecímu obvodu. Pokud je napětí připojeno, tak je 
tranzistor otevřen a baterie je nabíjena. Pokud napětí připojeno není, tak je tranzistor uzavřen a baterie je díky tomu odpojena od nabíjecího obodu. Díky tomu 
není vybíjena svodovým proudem pinu BAT. 

\section{Zvyšovač napětí pro LED}
Pro napájení vybraných inteligentních LED je zapotřebí napětí v rozsahu 4,5 až 5,5 V \cite{WS2812C_dtsh}. Použité baterie LiFePO4 mají napětí pouze 3,2 V, proto je 
zapotřebí použít zvyšovač napětí. 

Z komerčně dostupných integrovaných obvodů byl hledán zvyšovač napětí, který vytváří z napětí 3,3 V napětí 5 V a dodávat přitom do výstupu proud alespoň 200 mA. 
Maximální odběr všech 12ti potřebných inteligentní LED má maximální odběr 180 mA. S rezervou je tedy zapotřebí proud alespoň 200 mA. Nalezené obvody, které vyhovují 
těmto parametrům jsou LT1930 a MCP1640. 

Obvod LT1930 v doporučeném zapojení při vstupním napětí 3,3V vytváří výstupní napětí o hodnotě 5 V s maximálním odběrem proudu 480 mA \cite{LT1930_dtsh}. Napájecí napětí 
tohoto obvodu je v rozsahu 2,45 V až 16 V, což vyhovuje napájecímu napětí z baterií LiFePO4 \cite{LT1930_dtsh}.

%krátký popis MCP1640

%proč byl vybrán právě tento?
%Pro realizaci Semaforu byl zvolen obvod LT1930, který XXX protože XXX
%schéma zapojení vybraného typu

Pin /SHDN slouží k zapínání a vypínání obvodu. Pomocí přiloženého napětí 2,4 V a více na tento pin je obvod zapnut \cite{LT1930_dtsh}. Pin SW slouží pro  připojení cívky, 
případně diody, aby se snížilo elektromagnetické rušení \cite{LT1930_dtsh}. 

Pin FB slouží  pro zapojení zpětné vazby napětí na baterii. Jeho referenční napětí musí být nastaveno v rozmezí 1,240 V až 1,270 V, typická hodnota je však 1,255 V \cite{LT1930_dtsh}. 
Pro výstupní napětí 5 V byl zvolen rezistor R10 o hodnotě 13 k$\Omega$ z rezistorové řady E24 \cite{rezistorova_rada}. Řada E24 byla zvolena kvůli požadované přesnosti
napětí na pinu FB obvodu LT1930. Napětí na rezistoru R10 musí být tedy 1.255 V. Na rezistoru R9 je tedy úbytek napětí 3,745 V. Pomocí trojčlenky byla dopočítána hodnota 
rezistoru R9 dle rovnice:
\begin{equation} 
  R_{9}~=~\frac{R_{10}~\cdot~U_{R9}}{U_{R10}}~=~\frac{13~\cdot~3,745}{1,255}~=~38,79~\:k\Omega. 
  \quad
\label{eq:R9}
\end{equation}

Nejbližší hodnota rezistoru z rezistorové řady E24 je 39 k$\Omega$ \cite{rezistorova_rada}. Reálná hodnota napětí na rezistoru R10, tj. napětí na pinu FB byla dopočítána
dle rovnice:
\begin{equation} 
  U_{R10}~=~\frac{U_{OUT}}{R_{9}~+~R_{10}}~\cdot~R_{10}~=~\frac{5}{39~+~13}~\cdot~13~=~1,25~V. 
  \quad
\label{eq:UR10}
\end{equation}

Napětí 1,25 V je v povoleném rozmezí napětí na pinu FB. 

Přesné výstupní napětí se spočítá podle vzorce:
\begin{equation} 
  U_{OUT}~=~U_{FB}~\cdot~(1~+~\frac{R_{9}}{R_{10}})~=~1,25~\cdot~(1~+~\frac{39}{13})~=~5~V. 
  \quad \quad \quad \quad \cite{LT1930_dtsh}
\label{eq:VOUT_LT1930}
\end{equation}

%výběr diody
%výběr cívky

%moje schéma zapojení

\section{Konektor}
Jakoprogramovací konektor byl zvolen konektor USB typu C, který může být použit i jako konektor pro nabíjení baterie.

Tento konektor je v dnešní době velmi rozšířený a jeho použití se v následující době stále rozšiřuje. 

Není využíváno žádných výhod konektoru USB-C, jako je např. možnost power delivery apod. Je využíván pouze jako standardní a dostupný konektor, který je mezi běžnou
populací rozšířený a v následujících letech se bude rozšiřovat stále více. Je využito standardního jmenovitého napětí 5 V pro nabíjení baterií a nadále pinů D+ a D-, 
které jsou využity pro komunikaci při programování. 

Konektor USB-C je robustní a oboustranný, díky čemuž nebude docházet k tak častému poškození, jak by mohlo být např. u konektoru Micro USB. Při používání běžnou veřejností
se jedná o vítaný bonus. čš

Vybraný mikrokonrolér ESP32-C3 umožňuje komunikaci přímo po USB protokolu a není díky tomu zapotřebí žádného převodníku pro komunikaci \cite{ESP_C3_dtsh}. %jmenuje se to USB protokol?

%o transilech k USB + Shotkyny (celkový odběr zařízení z USB - shotkyny na správný proud)
%o rezistorech 5,1k

%ke každé komponentě napsat odběr

%na konci bude muset být blokové schéma konkrétních (vybraných) modulů

%power led pro první prototyp, potom nebudou, kvůli šetření energie, protože je to na baterky a mohlo by to při hrách hráče mást

%zvukova sinalizace - piezo
