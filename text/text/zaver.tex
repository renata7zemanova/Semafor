\chapter*{Závěr}
\phantomsection
\addcontentsline{toc}{chapter}{Závěr}

Tato práce se zabývá návrhem zařízení Semafor, které slouží jako doplněk pro táborové hry a také pro edukační účely. Byla navržena kompletní 
elektronika potřebná pro funkce, které má Semafor splňovat. 

V první části jsou rozebrány možnosti použitých komponent, které mohou být využity při výrobě Semaforu. Následně jsou z výběru použity 
nejvhodnější komponenty dle periferií, ceny i možnostech použití v outdoorových aplikacích.
Pro komunikaci Semaforů mezi sebou je připojen LoRa modul, který zajišťuje komunikaci na dostatečnou vzdálenost při použití na táborech. 

Výsledné schéma je přiloženo v příloze. 

Pro konfiguraci her, např. nastavení, která hra se hraje a s jakými parametry, slouží dotykové senzory a do budoucna bude také vytvořena 
webová stránka, ke které se bude připojovat pomocí WiFi. Konfigurace bude probíhat na vytvořené webové stránce. 