\chapter*{Závěr}
\phantomsection
\addcontentsline{toc}{chapter}{Závěr}
%přizpůsobit zadání 
Tato práce se zabývá návrhem zařízení Univerzálního modulu, které slouží jako doplněk pro outdoorové aktivity, a také pro edukační účely. Byla navržena kompletní 
elektronika potřebná pro funkce, které má Univerzální modul poskytovat. 

V~první části jsou rozebrány možnosti použitých komponent, které mohou být využity při výrobě Univerzálního modulu. Následně jsou z~výběru použity 
nejvhodnější komponenty dle periferií, ceny i možností použití v~outdoorových aplikacích. Pro bezdrátové nastavení byla zvolena konfigurace pomocí telefonu nebo 
notebooku. Proto byla vybrána WiFi technologie, která je v těchto zařízeních již zabudována. Konfigurace Univerzálního modulu pak probíhá na vytvořené webové stránce. 
Pro možnou komunikaci jednotlivých Univerzálních modulů mezi sebou je připojen LoRa modul E22-900T22D, který zajišťuje komunikaci na dostatečnou vzdálenost při použití 
v outdoorovém prostředí. 

Bylo vytvořeno schéma zapojení elektroniky pro Univerzální modul, které se skládá z~mikrokontroléru ESP32-C3 ve verzi s~anténou a LoRa modulu pro 
bezdrátovou komunikaci jednotlivých Univerzálních modulů mezi sebou. Napájení je realizováno pomocí baterie LiFePO4 a~USB-C konektoru pomocí powerbanky. Baterie 
jsou nabíjeny pomocí nabíjecího obvodu CN3058E. Komunikace s~okolním světem je zajištěna kapacitními tlačítky, 
vibračním motorem, piezem, fototranzistorem a programovatelnými LED typu WS2812C. Výsledné schéma zapojení elektroniky Univerzálního modulu je přiloženo v~příloze. 

Byla navržena, vyrobena a osazena DPS Univerzálního modulu, která se skládá ze 2 DPS spojených pomocí pinheadů. Na horní DPS jsou umístěna tlačítka, fototranzistor 
a programovatelné LED. Na spodní DPS je pak umístěna veškerá zbylé elektronika včetně baterií. 

Pro Univerzální modul byl také vytvořen základní firmware, který usnadňuje uživatelům programování tohoto zařízení. Byly také vytvořeny 3 herní módy, které slouží jako ukázka 
využití Univerzálního modulu. 

V poslední části je řešena voděodolnost Univerzálního modulu. Je vyřešena vyrobením krabičky na 3D tiskárně. DPS je do krabičky zalepena a její vrchní část je zalakována čirým
lakem. 
