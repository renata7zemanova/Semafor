\chapter*{Závěr}
\phantomsection
\addcontentsline{toc}{chapter}{Závěr}
%přizpůsobit zadání 
Tato práce se zabývá návrhem zařízení Semafor, které slouží jako doplněk pro táborové hry a také pro edukační účely. Byla navržena kompletní 
elektronika potřebná pro funkce, které má Semafor splňovat. 

V~první části jsou rozebrány možnosti použitých komponent, které mohou být využity při výrobě Semaforu. Následně jsou z~výběru použity 
nejvhodnější komponenty dle periferií, ceny i možností použití v~outdoorových aplikacích.
Pro komunikaci Semaforů mezi sebou je připojen LoRa modul, který zajišťuje komunikaci na dostatečnou vzdálenost při použití na táborech. 

Výsledné schéma je přiloženo v~příloze. 

Pro konfiguraci her, např. nastavení, která hra se hraje a s~jakými parametry, slouží kapacitní dotyková tlačítka a do budoucna bude také 
vytvořena webová stránka, ke které se bude připojovat pomocí WiFi. Konfigurace bude probíhat na vytvořené webové stránce. 

Bylo vytvořeno schéma zapojení elektroniky pro Semafor, které se skládá z~mikrokontroléru ESP32-C3 ve verzi s~anténou a LoRa modulu pro 
bezdrátovou komunikaci mezi jednotlivými Semafory. Napájení je realizováno pomocí baterie LiFePO4 a~USB konektoru pomocí powerbanky. Baterie 
jsou nabíjeny pomocí nabíjecího obvodu CN3058E a konektoru USB-C nebo USB-A. Komunikace s~okolním světem je zajištěna kapacitními tlačítky, 
vibračním motorem, piezem a inteligentními LED typu WS2812C. Mikrokontrolér ESP32-C3 nemá kapacitní vstupy, a proto je použit převodník AT42QT1070.
Pro napájení inteligentních LED je využit převodník napětí LT1930. 

