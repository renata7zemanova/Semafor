% V tomto souboru se nastavují téměř veškeré informace, proměnné mezi studenty:
% jméno, název práce, pohlaví atd.
% Tento soubor je SDÍLENÝ mezi textem práce a prezentací k obhajobě -- netřeba něco nastavovat na dvou místech.

\usepackage[
%%% Z následujících voleb jazyka lze použít pouze jednu
  czech-english,		% originální jazyk je čeština, překlad je anglicky (výchozí)
  %english-czech,	% originální jazyk je angličtina, překlad je česky
  %slovak-english,	% originální jazyk je slovenština, překlad je anglicky
  %english-slovak,	% originální jazyk je angličtina, překlad je slovensky
%
%%% Z následujících voleb typu práce lze použít pouze jednu
  %semestral,		  % semestrální práce (nesází se abstrakty, prohlášení, poděkování) (výchozí)
  %bachelor,			%	bakalářská práce
  master,			  % diplomová práce
  %treatise,			% pojednání o disertační práci
  %doctoral,			% disertační práce
%
%%% Z následujících voleb zarovnání objektů lze použít pouze jednu
%  left,				  % rovnice a popisky plovoucích objektů budou zarovnány vlevo
	center,			    % rovnice a popisky plovoucích objektů budou zarovnány na střed (vychozi)
%
]{thesis}   % Balíček pro sazbu studentských prací

\usepackage{microtype}

%%% Jméno a příjmení autora ve tvaru
%  [tituly před jménem]{Křestní}{Příjmení}[tituly za jménem]
% Pokud osoba nemá titul před/za jménem, smažte celý řetězec '[...]'
\author[Bc.]{Renata}{Zemanová}

%%% Identifikační číslo autora (VUT ID)
\butid{211251}

%%% Pohlaví autora/autorky
% (nepoužije se ve variantě english-czech ani english-slovak)
% Číselná hodnota: 1...žena, 0...muž
\gender{1}

%%% Jméno a příjmení vedoucího/školitele včetně titulů
%  [tituly před jménem]{Křestní}{Příjmení}[tituly za jménem]
% Pokud osoba nemá titul před/za jménem, smažte celý řetězec '[...]'
\advisor[doc.\ Ing.]{Pavel}{Šteffan}[Ph.D.]

%%% Jméno a příjmení oponenta včetně titulů
%  [tituly před jménem]{Křestní}{Příjmení}[tituly za jménem]
% Pokud osoba nemá titul před/za jménem, smažte celý řetězec '[...]'
% Nastavení oponenta se uplatní pouze v prezentaci k obhajobě;
% v případě, že nechcete, aby se na titulním snímku prezentace zobrazoval oponent, pouze příkaz zakomentujte;
% u obhajoby semestrální práce se oponent nezobrazuje (jelikož neexistuje)
% U dizertační práce jsou typicky dva až tři oponenti. Pokud je chcete mít na titulním slajdu, prosím ručně odkomentujte a upravte jejich jména v definici "VUT title page" v souboru thesis.sty.
\opponent[doc.\ Mgr.]{Křestní}{Příjmení}[Ph.D.]

%%% Název práce
%  Parametr ve složených závorkách {} je název v originálním jazyce,
%  parametr v hranatých závorkách [] je překlad (podle toho jaký je originální jazyk).
%  V případě, že název Vaší práce je dlouhý a nevleze se celý do zápatí prezentace, použijte příkaz
%  \def\insertshorttitle{Zkác.\ náz.\ práce}
%  kde jako parametr vyplníte zkrácený název. Pokud nechcete zkracovat název, budete muset předefinovat,
%  jak se vytváří patička slidu. Viz odkaz: https://bit.ly/3EJTp5A
\title[Universal module for supporting team games]{Univerzální modul pro podporu týmových her}

%%% Označení oboru studia
%  Parametr ve složených závorkách {} je název oboru v originálním jazyce,
%  parametr v hranatých závorkách [] je překlad
\specialization[Microelectronics]{Mikroelektronika}

%%% Označení ústavu
%  Parametr ve složených závorkách {} je název ústavu v originálním jazyce,
%  parametr v hranatých závorkách [] je překlad
%\department[Department of Control and Instrumentation]{Ústav automatizace a měřicí techniky}
%\department[Department of Biomedical Engineering]{Ústav biomedicínského inženýrství}
%\department[Department of Electrical Power Engineering]{Ústav elektroenergetiky}
%\department[Department of Electrical and Electronic Technology]{Ústav elektrotechnologie}
%\department[Department of Physics]{Ústav fyziky}
%\department[Department of Foreign Languages]{Ústav jazyků}
%\department[Department of Mathematics]{Ústav matematiky}
\department[Department of Microelectronics]{Ústav mikroelektroniky}
%\department[Department of Radio Electronics]{Ústav radioelektroniky}
%\department[Department of Theoretical and Experimental Electrical Engineering]{Ústav teoretické a experimentální elektrotechniky}
%\department[Department of Telecommunications]{Ústav telekomunikací}
%\department[Department of Power Electrical and Electronic Engineering]{Ústav výkonové elektrotechniky a elektroniky}

%%% Označení fakulty
%  Parametr ve složených závorkách {} je název fakulty v originálním jazyce,
%  parametr v hranatých závorkách [] je překlad
%\faculty[Faculty of Architecture]{Fakulta architektury}
\faculty[Faculty of Electrical Engineering and~Communication]{Fakulta elektrotechniky a~komunikačních technologií}
%\faculty[Faculty of Chemistry]{Fakulta chemická}
%\faculty[Faculty of Information Technology]{Fakulta informačních technologií}
%\faculty[Faculty of Business and Management]{Fakulta podnikatelská}
%\faculty[Faculty of Civil Engineering]{Fakulta stavební}
%\faculty[Faculty of Mechanical Engineering]{Fakulta strojního inženýrství}
%\faculty[Faculty of Fine Arts]{Fakulta výtvarných umění}
%
%Nastavení logotypu (v hranatych zavorkach zkracene logo, ve slozenych plne):
\facultylogo[logo/FEKT_zkratka_barevne_PANTONE_CZ]{logo/UTKO_color_PANTONE_CZ}

%%% Rok odevzdání práce
\graduateyear{2023}
%%% Akademický rok odevzdání práce
\academicyear{2022/23}

%%% Datum obhajoby (uplatní se pouze v prezentaci k obhajobě)
\date{11.\,01.\,2023} 

%%% Místo obhajoby
% Na titulních stránkách bude automaticky vysázeno VELKÝMI písmeny (pokud tyto stránky sází šablona)
\city{Brno}

%%% Abstrakt
\abstract[%
The goal of this diploma thesis is to design device called Universal module, which serves as an assistant in outdoor games. The design focuses 
on safety, simplicity and low cost. 
%dopsat
This thesis deals with the selection and design of the overall electronics contained in the Traffic light. Emphasis is placed on the selection 
of the light signaling, wireless module and microcontroller.
]{%
Cílem práce je navrhnout elektronické zařízení Univerzální modul, který slouží jako pomocník při týmových outdoorových hrách. 
Při návrhu je kladen důraz na bezpečnost, jednoduchost a nízkou cenu. Kvůli outdoorovému použití je řešena také voděodolnost.

Tato práce se zabývá výběrem a návrhem elektroniky. Je kladen důraz na výběr světelné signalizace, bezdrátového modulu, napájení a mikrokontroléru.
Návrh DPS probíhal na 2 spojených deskách. Byl vytvořen FW pro zjednodušené programování Univerzálního modulu a vznikly také demonstrační módy 
pro různé aktivity. 
}

%%% Klíčová slova
\keywrds[%
Universal module, microcontroller, programmable LED WS2812C, batteries LiFePO4, LoRa module, capacitive touch buttons, silicone
]{%
Univerzální modul, mikrokontrolér, programovatelné LED WS2812C, baterie LiFePO4, LoRa modul, kapacitní dotyková tlačítka, silikon
}

%%% Poděkování
\acknowledgement{%
Ráda bych poděkovala vedoucímu diplomové práce panu doc. Ing.~Pavlovi Šteffanovi, Ph.D.\ za odborné vedení,
konzultace, podnětné návrhy k~práci a zapůjčení testovacího hardwaru. Dále také děkuji RNDr. Janovi Mrázkovi 
a Jakubovi Andrýskovi za rady při tvorbě firmwaru a při tvorbě silikonového obalu. Dík také patří Ing. Jakubovi 
Streitovi za rady při návrhu elektroniky.  
}%